%
%  \item{用語名}{ふりがな}{解説}
%   「ふりがな」は現在使ってませんが、項目の自動ソート化を
%   予定しているので、新規追加項目では記入してください。
%   「ふりがな」には、アルファベット(半角英数・半角記号)は
%   そのまま、漢字・カタカナはひらがなに変えて記入。
%
\begin{用語集}
\section{A}
\item{ABINIT-MP(X)}{}
{望月・中野らが開発している国産FMO計算用のプログラム
システム。4体のフラグメント展開までが可能。}
\item{AFO}{}
{FMO計算を2体展開の範囲内で固体系に適用するための技法の1つ。}
\item{Allreduce QR法}{}
{QR分解の手法の一種。高並列環境で高い並列性能を実現するために近年提案された新しい手法。}
\item{allreduce, allgather, alltoall}{}
{計算ノード間での集団通信の様式。接頭辞のallはすべてのノードに渡って結果を共有する事を示す。allreduceは総和などリダクション型の演算を計算ノードにまたがって行うこと。allgatherはすべてのノードから部分データを集めて全体データを作ること。alltoallは全体全通信によって、データの転置を行うこと。}
\item{AU}{}
{天文単位(Astronomical Unit)の略称。天文単位とは長さの単位であり、地球と太陽間の距離に由来する。今日では149,597,870,700メートルと定義される。}
\section{B}
\item{Bi-CGSTAB法}{}
{疎行列の連立一次方程式の解法の一つ。反復解法の一つである双共益勾配法(BiCG法)の残差を減少させ、安定化した手法。比較的高速・安定とされる手法で、偏微分方程式を解く多くの物理問題(熱流体・構造・電磁場など)の核となるソルバとして広く使用されている。}
\item{Bisection BW}{}
{バイセクションネットワークバンド幅のこと}
\item{BLAS}{}
{Basic Linear Algebra Subprograms(BLAS)。ベクトルと行列に関する基礎的な線型代数演算のサブプログラム集のこと。線形代数演算ライブラリAPIのデファクトスタンダードでもあり、高度に最適化された実装がインテルなどの各ハードウェアベンダーなどから提供されている。}
\item{B中間子}{}
{ボトムクォークを含む中間子の総称。}
\section{C}
\item{CP対称性}{}
{Cは荷電共役変換(粒子$\Leftrightarrow$反粒子)、Pはパリティ変換(鏡映変換)を表す。これらの変換の組み合わせによって理論が不変であるとき、その理論はCP対称性を持つという。}
\section{D}
\item{DGEMM}{}
{倍精度汎用行列乗算のためのBLASのサブルーチンのひとつ。LINPACKベンチマーク内で多用されているため、DGEMM実装の性能はベンチマーク結果に大きな影響を与える。}
\item{DZP基底関数}{}
{double zeta polarization基底関数。1s, 2pなどの原子基底の各成分を、二つの短縮ガウス型関数を用いて表現した基底をDZ(double zeta)基底と言い、それに分極関数を加えたもの。}
\section{E}
\item{EDA標準技術}{}
{電子情報技術産業協会 EDA標準技術専門委員会配下のEDA標準化小委員会において推進している国際標準化機構の活動に対応したEDA(Electronic Design Automation)の標準化活動のこと。}
\item{ESR}{}
{電子スピン共鳴の略。開殻系の電子状態に関する情報が得られる。}
\section{F}
\item{Fermi-Pasta-Ulamの非線形励起}{}
{非線形なバネで互いにつながれた多数の粒子の運動において孤立した波が生じる現象。ソリトンと呼ばれる}
\item{Finite-difference Time-domain法}{}
{電磁場解析等で用いられる計算手法の一つ。空間を差分近似し陽的な時間進行法を用いる。}
\item{FMO}{}
{フラグメント分子軌道法(FMO法: Fragment Molecular Orbital Method)は、北浦和夫教授(現神戸大学)によって開発された量子化学理論。分子全体を小さなフラグメントに分割して計算をするため、通常の量子化学計算では不可能なタンパク質のような大規模分子系の量子化学計算が実行可能。また、分割した小規模のフラグメントごとに並列計算を実行することが可能なため、非常に効率よく並列計算を実行可能。}
\item{Fock行列}{}
{多数の電子の振る舞いを平均化されたポテンシャル中を動く、一電子のシュレディンガー方程式を行列表現した行列のこと。}
\section{G}
\item{GAMESS}{}
{米ゴードン研がDOEなどの資金で開発を続けている汎用の分子軌道計算プログラム。10年ほど前からFMO計算機能が導入されている。}
\item{gatherv}{}
{MPIのデータ転送関数(MPI\_GATHERV)。全てのプロセスから一つの宛先プロセスにメッセージを転送する。}
\item{Gauss関数の局所性}{}
{中心からの距離が大きくなるにつれて急速に値が減衰するGauss関数の性質。2電子クーロン反発積分などの計算で、これを利用したカットオフ、演算削減は大きな効果がある。}
\item{GCC}{}
{GNU Compiler Collection.自由に使えるC/C++言語などのプログラム言語のコンパイラ.}
\item{(gg{\textbar}gg)型積分}{}
{2電子クーロン反発積分は4つの基底関数中心を持つが、その4つの基底関数ともに全角運動量5のg軌道関数を含む2電子クーロン反発積分のこと。}
\item{GMRES法}{}
{疎行列の連立一次方程式の解法の一つ。反復解法の一種で、比較的ロバストなクリロフ部分空間法の一つとして知られている解法。同時に使用する前処理法や計算条件によって、並列計算性能が高く、高速に収束解が得られるため、偏微分方程式を解く物理問題の一部で使用されている。}
\item{Gタンパク質共役受容体}{}
{細胞外の神経伝達物質やホルモンを受容してそのシグナルを細胞内に伝える受容体。その際Gタンパク質と呼ばれる三量体タンパクを介してシグナル伝達が行われる。多くの薬剤のターゲットになっている。}
\section{H}
\item{HF交換相互作用}{}
{密度汎関数法において、交換相互作用を表現する汎関数にHartree-Fock(HF)法の交換相互作用を使ったもの}
\item{High-radix型}{}
{ある計算ノードからもう一つの計算ノードへの通信が、その他の計算ノード同士の間の通信と同時に実行しやすいネットワーク。すなわち他のノードに妨害されずに通信できる一ノードあたりのノード数が多いネットワーク。}
\item{Hodgkin-Huxley formalism}{}
{イカ巨大軸索を対象に神経細胞における活動電位の発生メカニズムをゲート(後に実体としてこれに相当するイオンチャネルがあることが明らかにされた)の協同性を用いて電気回路として記述したモデルがHodgkin-Huxleyモデルである。この形式をHodgkin-Huxley formalismと呼び、多くの種類のイオンチャネルやマルチコンパートメントモデルに対しても用いられる。}
\item{HPCI戦略分野}{}
{スーパーコンピュータ「京」を中心としたHPCI(High Performance Computing Infrastructure)を最大限に活用することによって、戦略的に取り組むべき5つの研究分野}
\section{I}
\item{IACM}{}
{International Association for Computational Mechanics。国際計算力学連合。計算力学に関するいくつかの国際学術講演会を運営する。}
\section{L}
\item{L1,L2キャッシュ}{}
{CPUにはメモリとのデータ転送を節約するためのデータの一時的な保管場所があり、それをキャッシュという。キャッシュは演算装置とメモリとの間に多階層に配置されており、演算装置に近い順にL1, L2と言う。}
\item{L1正則化法}{}
{影響を与える因子の数を抑えることができる機械学習法}
\item{Langevin方程式}{}
{ブラウン運動を記述する確率微分方程式}
\item{Leaky Integrate-and-Fire neuronモデル}{}
{積分発火モデルの中でリーク電流をつけたモデル}
\item{Lennard-Jones型ポテンシャル}{}
{原子間の相互作用を記述する最も簡単なポテンシャルで、斥力の項と引力の項の2つによって記述される。}
\item{LHC}{}
{大型ハドロン衝突型加速器(Large Hadron Collider)。欧州原子核研究機構(CERN)で稼働中の加速器の名称。ヒッグス粒子の発見と超対称性粒子などの新しい物理の探索を目指している。}
\item{logP}{}
{化合物の脂溶性を表す量}
\item{LPB}{}
{LSI Package Boardの略。}
\section{M}
\item{MIPSピーク性能}{}
{MIPS(ミプス)値、あるいはMIPSピーク性能は、100万命令毎秒 (million instructions per second) の略で、コンピュータの性能指標の1つ。}
\item{MPI}{}
{並列計算のためのプロセス間通信ライブラリの業界標準}
\section{N}
\item{NMR分光}{}
{原子核の磁気を測定する手法。分子構造に関するデータが得られる。}
\item{NP完全問題}{}
{クラスNPに属する問題でかつ、クラスNPすべての問題から多項式時間帰着可能な問題。このクラスに属する問題は多項式時間で解を見つけるアルゴリズムが存在しないと予想されている。(P${\neq}$NP予想)}
\section{O}
\item{on the fly}{}
{「実行中に」を意味し、プログラム中で繰り返し必要となるデータを、その度ごとに計算して用いるアルゴリズムを指す。これと対極にあるのは「あらかじめ計算して保存しておいたデータを、必要になる度に記憶装置から参照して用いる」やり方である。}
\item{ONIOM法}{}
{ONIOM法は諸熊啓治教授(現京都大学)により考案されたQM/MM計算の代表的な方法。生体高分子などの巨大分子をいくつかのレイヤーに分け、レイヤーごとに量子化学計算や分子力学計算を行うことで、巨大分子の電子状態や分子構造の評価や反応機構の解析を行うことが可能。}
\section{P}
\item{PDB構造}{}
{Protein Data Bank(PDB)に登録されているNMR解析やX線構造解析などの実験的手法によって得られた蛋白質の構造。}
\item{pKa}{}
{化合物の酸性度を表す量}
\section{Q}
\item{QueryDriven}{}
{データに対するクエリー(質問)を行いながら,対話的にデータを調べていく手法.}
\section{R}
\item{RI(放射性同位元素)ビームファクトリー}{}
{安定な原子核に含まれる中性子の数よりも、中性子の数がかけ離れた原子核をエキゾチック原子核という。このようなエキゾチック原子核は不安定なため天然には存在しない。しかし超新星爆発などの高エネルギー現象による重元素合成では、エキゾチック原子核が中性子過剰核として重元素合成の反応経路にあらわれ重要な役割を果たす。このようなエキゾチック原子核の性質を解明することは原子核理論のチャレンジであるが、実験的にも生成が難しい。このようなエキゾチック原子核の性質を調べることのできる実験設備が理研の所有するRIビームファクトリーである。エキゾチック原子核の生成率は低いため大強度のビームが必要である。理研RIビームファクトリーは2006年から稼働しており、世界最強のビーム強度を誇り、これまでさまざまな新しいエキゾチック原子核を発見している。米国および独国においてもそれぞれ2018年と2016年の稼働を目指してより大強度のRIビームファクトリーの計画がある。}
\item{r過程}{}
{宇宙における重元素生成過程は主に、星の内部で安定線上を時間をかけて進む中性子捕獲反応(s過程)と、わずか数秒間の爆発的な過程で安定線から離れた原子核を作るr過程に分けられる。図4.5.3.1も参照。}
\section{S}
\item{SPICEモデル}{}
{SPICE (Simulation Program with Integrated Circuit Emphasis)はカリフォルニア大学バークレー校で開発された回路シミュレータであり、SPICEモデルとはこの回路シミュレータで使用される、受動素子(抵抗、インダクタ、コンデンサ等)と能動素子(トランジスタ等)の等価回路モデルのこと。}
\item{STM}{}
{走査型トンネル顕微鏡。短針と固体側とのトンネル電流の観測により、表面構造や電子状態を知ることが可能。原理は異なるが、原子間力顕微鏡(AFM)なども表面解析に用いられる。}
\item{Structure-based drug design (SBDD)}{}
{タンパク質などの立体構造をもとにして薬剤のデザインをする方法}
\item{SU(3)群}{}
{ゲージ群の一つ。QCDはSU(3)ゲージ群のゲージ理論である。}
\item{SuperKEKB/BelleII実験計画}{}
{B中間子の寿命やB中間子がどのような粒子にどのような割合で崩壊するかを精密に測定する実験。高エネルギー加速器研究機構で行われている。2021年頃に高精度データが取得できるよう計画されている。B中間子に含まれるボトムクォークは、トップクォークに次ぐ質量を持つ重いクォークであり、ボトムクォークの性質を詳細に調べることで、素粒子標準理論に内在する階層性の起源や素粒子標準理論にない新しい物理を明らかにできる。}
\section{T}
\item{tagSNP}{}
{ゲノムの特定の領域においてその領域中の他のSNP(一塩基多型)の代表となりうるSNP}
\item{Thin node}{}
{少数の演算器、メモリによって構成される計算ノード。}
\section{U}
\item{udsハドロン}{}
{6種類のクォークのうち、軽いu,d,sの3種類のクォークから構成されるハドロン。}
\item{union-findアルゴリズム}{}
{グラフ構造の中から、連結クラスター(互いに辺でつながっている頂点の集合)を見つけ出す際に用いられるアルゴリズム。}
\section{V}
\item{VOF関数}{}
{Volume Of Fluidの略。空間を計算要素に分割した際、その計算要素に占める流体の体積比率を用いる手法をVOF法と呼ぶ。このとき使用する全計算要素のVOF値の事をVOF関数と呼ぶ。}
\section{X}
\item{XFEL}{}
{X線自由電子レーザー(XFEL)は,波の位相がきれいにそろったレーザーの性質を持つ超高輝度のX線を発生させることのできる光源}
\section{あ}
\item{アーティフィシャルニューラルネットワークモデル}{}
{複数の同種神経の平均としての活動量関数とシナプス伝達関数を定義してネットワークを形成させるモデル。広義のMcCulloch-Pitts Model }
\item{足場タンパク質複合体}{}
{細胞内情報伝達系において、複数の情報伝達タンパク質と結合して複合体を形成する足場となるタンパク質の複合体}
\item{アパタイト}{}
{リン酸カルシウム(燐灰石)のことだが、生体では水酸基が入ったヒドロキシアパタイトとして歯や骨の主要構成要素となっている。}
\item{アルダー転移}{}
{剛体球の密度を上げると、ある密度を境に液体から固体(結晶)に相転移すること。}
\item{アンサンブルシミュレーション}{}
{沢山のシミュレーションを行い、その統計的性質を研究する計算手法}
\item{アンジュレータ}{}
{加速された電子の直線軌道上に沿って、多数のN、Sの磁極からなる磁石列を上下に配置して、その間を通り抜ける電子を周期的に小さく蛇行させて、明るく特定の波長を持った光を作り出す装置。}
\section{い}
\item{位相空間}{}
{燃料プラズマ粒子の3次元位置と3次元速度を座標とする6次元空間。粒子間の衝突効果が十分に大きければ、局所的な熱力学的平衡を仮定して3次元流体モデルでプラズマを記述できるが、衝突効果が小さい高温プラズマに対しては6次元位相空間の粒子分布を記述する運動論モデルが必要になる。ただし、磁場閉じ込め核融合プラズマのような強磁場中の運動論モデルは5次元位相空間に簡約化できる。}
\item{位相空間密度}{}
{位相空間における密度。位相空間とは位置と速度(または運動量)を座標とした空間のことである。例えば、我々の世界ではそれぞれ3次元で合せて6次元の空間。}
\item{位相骨格}{}
{データを変化点の接続情報(スケルトン・骨格)により表し,大規模なデータを非常に小さなデータサイズで特徴付けることができる.}
\item{1磁場散逸時間}{}
{磁場を作っている電流が電気抵抗により熱に変わること(ジュール散逸)によって、磁場が指数関数的に減少する典型的時間。}
\item{一般相対性理論}{}
{アインシュタインによって提案された重力の理論。物質のエネルギーが時空の幾何学を決定する理論。時空の幾何学を重力とみなす。星の重力を決定するだけでなく、宇宙全体の幾何学をも決定でき、宇宙物理学における基礎となる理論。量子力学が重要となるミクロの世界での重力の振る舞いについては記述できない。}
\item{遺伝子プロモータ}{}
{特定の遺伝子の発現を促すタンパクなどの細胞内物質}
\item{イベント駆動型}{}
{現象や手順を有限の数の瞬間的に起きる事象(イベント)の連続として扱うやり方}
\item{陰解法}{}
{時間積分の一つ。時間微分の離散化において後退差分(現在と過去の値を使って離散化する)を用いて離散化を行う手法。元の偏微分方程式は未知変数の連立一次方程式にと離散化され、この連立一次方程式を解くことになる。}
\item{インフレーション}{}
{宇宙誕生直後における宇宙の指数関数的膨張のこと。}
\section{う}
\item{ウィーク・スケール}{}
{並列単位当たりの問題サイズを一定にして、並列数を増やしていく場合(つまり、問題サイズが並列数に比例して大きくなる)での、計算時間の変化}
\section{え}
\item{エキゾチックハドロン}{}
{中間子(メソン)はクォークと反クォークから構成され、重粒子(バリオン)は3個のクォークから構成されると考える単純なクォークモデルからは予測できない異種のハドロン。}
\item{液体論}{}
{液体は気体に比べて原子、または分子間の相互作用が強く、また固体とは違いこれらの粒子が動き回るためその取扱いは容易ではない。液体そのものから溶媒としての性質等についても議論がされており、数値シミュレーションによる研究も盛んである。}
\item{エネルギースケールの階層性}{}
{素粒子標準理論のパラメータには以下のような階層性がある。 (1) 弱い力の媒介粒子の質量が重い。(陽子の約80倍と約90倍) クォークやレプトンの質量はバラバラであるが、 (2) トップクォークは特に重い。(陽子の約170倍) (3) ニュートリノの質量が直接測定不可能なぐらい軽い。 質量はエネルギーと等価であるので、エネルギースケールの階層性という。このような階層性の起源を明らかにすることは現在の素粒子物理学の課題である。}
\item{エネルギー分散外挿法}{}
{通常、変分計算によって得られたエネルギー期待値は、真のエネルギー期待値の上限しか与えることができない。変分空間を徐々に広げて、エネルギー期待値をエネルギー分散期待値の関数として外挿することによって、精度よく真のエネルギー期待値を見積もる方法。}
\item{エピジェネティクス}{}
{DNA塩基配列の変化を伴わないが、細胞分裂後も継承される遺伝子発現あるいは細胞表現型を研究する学問領域。}
\item{円偏光}{}
{光を始めとした電磁波は、進行方向と垂直に電場と磁場が振動する横波である。円偏光では、進行方向と直交する平面上で電場もしくは磁場の向きが円運動を描く。進行方向を手前に取って時計回り、反時計回りのものが存在する。}
\section{お}
\item{オーダリング}{}
{主に、メモリ空間でのデータの連続性を改善するために、数値データの格納順序を入れ替え、計算機による処理性能の向上を図ることを指す。}
\section{か}
\item{カーテシアン座標系}{}
{直交座標系の事。空間の位置を示すのに互いに直交する座標系を用いる。}
\item{カーネル最適化技術}{}
{プログラムにおいて主要なコストを占める逐次演算処理をカーネルと呼ぶ。プロセッサのアーキテクチャに依存して逐次演算処理の最適化方法は異なるため、特に、メニーコアプロセッサを効率的に利用するには新たな最適化技術の開発が必要となる。}
\item{カーパリネロ法}{}
{電子状態計算により原子にかかる力を直接見積もりながら、分子動力学計算を行う手法の1つ。電子状態に時間発展方程式を導入し、計算の高速化を実現している。}
\item{階層的時ステップ}{}
{要素により計算時ステップ幅に幅がある場合に例えば2の整数乗などのあらかじめ決めた規則に沿って時ステップ幅を決定する事で同期を容易にするやり方}
\item{カイラル凝縮}{}
{クォーク・反クォーク対が凝縮し、真空期待値を持つこと。}
\item{カイラルなゲージ対称性}{}
{フェルミオンの右巻き成分と左巻き成分が異なるゲージ対称性を持つ場合の対称性のこと}
\item{カイラル有効場理論}{}
{低エネルギー領域における物理現象を記述するために必要な自由度だけを取り入れた近似的な理論。}
\item{カイラル対称性}{}
{質量ゼロのフェルミオンが持つ対称性の一つ。光速で運動するフェルミオンはそのスピンが運動量に対して平行な場合(右巻き)と反平行な場合(左巻き)の2つの独立な自由度に分かれる。理論が、右巻き粒子だけで(または左巻き粒子だけで)、粒子の入れ替え操作に対し不変である場合にカイラル対称性があるという。標準理論は質量ゼロのフェルミオンによって構成されている。}
\item{核子}{}
{原子核を構成する陽子と中性子の総称。大きさはおよそ$10^{-15}$m。核子は3個のクォークが強い力で結合した粒子である。}
\item{核図表、安定線、エキゾチック核}{}
{原子核は陽子と中性子から構成されている。陽子数・中性子数をそれぞれ縦軸・横軸にとってこれを平面的に図示したものが核図表(nuclear chart)。この核図表上で、自然界に存在する安定な原子核は1次元の線のようになるため、安定線と呼ぶ。この安定線から離れた原子核は有限の寿命で崩壊するが、陽子数と中性子数が大きくことなる原子核も存在し、ここではこれらをエキゾチック核と呼んでいる。}
\item{核変換テクノロジー}{}
{原子炉の廃棄物処理の一つとして、長い寿命をもつ放射性同位元素や特に毒性の強く危険なものを、核反応を利用して短い寿命のものに変換させ消滅させるために必要な技術、方法、基礎知識等。}
\item{核力}{}
{核子やバリオンの間に働く力。陽子と中性子を結び付けて原子核を形作る。湯川秀樹博士は核力をパイ中間子の交換による作用であると提唱し、実際にパイ中間子が発見された。核力は基礎的な力でなく強い力による副次的な力であり、複雑な様相を呈する。たとえば、3つの核子の間に働く核力(3体力)は2つの核子間に働く核力(2体力)の単純な重ね合わせではないことが挙げられる。核力の性質の理解には、強い力の深い理解が必要である。}
\item{過減衰極限}{}
{Langevin方程式において慣性力を無視できるとした場合の特殊ケース}
\item{火成活動}{}
{マグマの発生や移動に伴って生じる諸現象の総称。}
\item{可塑的触媒場}{}
{タンパク質の構造変化などにより変化可能な触媒を行う環境}
\item{カットオフ半径}{}
{短距離の成分のみを取り扱う場合に、どの程度の長さまで扱うかという距離。この距離より離れた成分は0と考える。}
\item{カラーグラス凝縮}{}
{高エネルギーのハドロンで、量子色力学で「色」をもつグルーオンが大量に生成されて高密度に凝縮した状態。}
\item{軽い原子核、重い原子核}{}
{陽子数と中性子数の和を質量数と呼び、原子核の質量はほぼ質量数に比例する。「軽い」「重い」とは、この質量数の大きさを指している。明確な線引きはできないが、質量数が10程度以下のものは軽い原子核、100に近くなると重い原子核と呼ばれる。}
\item{カルシウムイメージング}{}
{カルシウム感受性蛍光色素を標的細胞に導入して、蛍光観察を行う方法。一般にカルシウムの配位結合によるセンサー分子のコンフォメーション変化は大きく蛍光変化も大きい。そのせいか蛍光プローブを使った神経活動観察法としてはカルシウムイメージングは主流でありつづけている。脳組織内の多点同時観察を見据えると蛍光プローブの導入法が重要で特定神経組織へのローカルインジェクションや特定の遺伝子プロモータを標的としたカルシウムセンサータンパク質の遺伝子導入が2000年代になって多く行われている。}
\section{き}
\item{季節内振動現象}{}
{中緯度帯にみられる高低気圧等の総観規模現象に比べて長く、季節変化より短い時間スケール(おおよそ10日~90日周期)の現象を総称して季節内振動現象と呼ぶ。有名な季節内振動現象として、地球規模の活発な積雲活動域が熱帯を東進していくMadden-Julian振動や、アジアにおけるモンスーン活動が知られており、中長期予報を行う際の重要な現象と考えられている。}
\item{基底重なり}{}
{電子雲を表現するために用いられる局在基底もしくは平面波基底間の空間的なオーバラップ(重なり)のこと。異なった平面波基底間の重なりは全空間で積分をするとゼロとなるが、局在基底間では重なり積分はゼロでない場合がある。}
\item{基底関数極限}{}
{無限に多くの基底関数を用い、基底関数展開による誤差がなくなる極限。デジタルカメラの画素数が上がり、アナログ写真との差がなくなった極限のような概念。}
\item{逆引き用分割テーブル}{}
{配列の添字から対応する電子・スピンの状態を求めるためのテーブル。部分系に分割したテーブルを組み合わせて用いることで、そのサイズを大幅に小さくすることが可能となる。}
\item{ギャザー・スキャッタ機構}{}
{配列に対する間接インデックス参照を効率的に行うためのハードウェア組み込み機構。}
\item{キャビテーション}{}
{液体の流れの中で局所的に圧力が変化することにより短時間に泡の発生と消滅が起きる物理現象であり空洞現象とも言われる。キャビテーションの発生は、発生する気泡により、ポンプなどの流体機器における振動・騒音の発生や性能低下の原因となる。また同時に発生する圧力波がこれらの機器表面のエロージョン(壊食)を起こして、効率を下げたり破壊することがある。}
\item{球面調和関数展開}{}
{球面調和関数は完全性をもち、球面上の任意の連続関数を一意に展開できる。このため、球面上のスカラー場の表現に用いられる。}
\item{境界埋込法}{}
{流体の運動と構造体の変形を同時(連成問題)にシミュレーションするときに用いる手法。流体をオイラー座標系で表現し、構造物をラグランジュ座標系で表現する。}
\item{虚時間軸}{}
{ある温度における統計力学的な平衡状態をあらわす式が、見た目上、通常の量子力学的な時間発展の式の「時間」のところに純虚数の値を入れた形になっており、「虚時間」と呼ばれます。単に見た目の問題というだけではなく、実時間$\Leftrightarrow$虚時間の対応を考えることにより理論的にも見通しが良くなることが多い。}
\item{強震動}{}
{明確な定義を持つ言葉ではないが、一般に、建築・土木構造物の被害に直接関与するような地表面での強い地震動のことをいう。}
\item{共発現解析}{}
{ある遺伝子の発現と相関の高い遺伝子を同定し特定の生物学的現象に互いに関係のある遺伝子群の機能などを解析する方法}
\item{共役勾配法}{}
{連立一次方程式を解くため、または制限付きの2次形式の極値を求めるための反復的アルゴリズムの一つ。}
\item{共溶媒濃度}{}
{溶液中の溶質および主なる溶媒のほかに含まれる第二の溶媒成分の濃度}
\item{行列模型}{}
{弦理論の非摂動的定式化の一つ。}
\item{局在基底}{}
{量子力学的には電子は点ではなく雲のように広がっている。この広がりを表し電子の雲の状態を記述するために用いられる関数のこと。電子は周囲の環境により、電子雲の広がりかたの度合いはことなるが、特にその広がりが強くない場合に用いられる関数のことを局在基底と呼ぶ。分子・原子では、電子の広がりは限定的であるために、原子・分子中の電子雲の状態を記述するために局在基底はしばしば用いられる。}
\item{局在軌道}{}
{特定の原子あるいは結合領域に、空間的に局在した分子軌道のこと。分子の量子化学計算で得られる分子軌道は、通常、分子全体に広がった(非局在化)形状をしているが、これらの非局在分子軌道に特定のユニタリー変換を施すことによって、局在軌道に変換することが出来る。空間的に離れた局在軌道どうしの積は無視できるほど小さくなることを利用して、計算コストの軽減をはかることができるほか、計算結果の物理化学的な解釈を手助けする目的にも用いられる。}
\item{局所準粒子乱雑位相近似}{}
{量子多体系において、非平衡状態の規準モード(近似的に独立な運動)を決定する方法。}
\item{巨大応答}{}
{外部からの磁場や電場や光照射などの刺激によって物質中の集団秩序を変化させ、抵抗値などを劇的に変化させること。}
\item{均質化法}{}
{マルチスケール解析手法の一つ。材料の詳細ミクロ構造をマクロ解析に反映させるために、ミクロとマクロの連成解析を行う。}
\item{金属原子拡散}{}
{燃料極における多孔質構造を変化させる金属原子の移動.三相界面長さの減少を通じて反応性を低下させる.}
\item{金属誘起ギャップ状態}{}
{半導体と金属の界面において金属の電子状態が半導体にしみ出すことで半導体ギャップ中に生成される新たな電子状態・準位。}
\section{く}
\item{クォーク}{}
{アップ、ダウン、チャーム、ストレンジ、トップ、ボトム、と名付けられた質量の異なる6種類のフェルミオンの族名。電磁気力、弱い力、強い力を受ける。アップクォークとダウンクォークは強い力により束縛しあい、陽子や中性子、中間子などの粒子を形成する。クォークの名前の違いは質量によって決まっており、質量の軽い順にクォークを並べると、アップ、ダウン、ストレンジ、チャーム、ボトム、トップとなる。}
\item{クォーク・グルーオン・プラズマ}{}
{通常、クォークはハドロンの中に閉じ込められているが、高エネルギー状態では自由に動き回れるようになる。クォークとグルーオンが電離したプラズマ状態。}
\item{クォーク・グルーオンプラズマ相}{}
{クォークはハドロンの中に閉じ込められておらず、自由に動き回れる状態。}
\item{クォーク作用}{}
{クォークに対する作用。作用から運動方程式などが得られる。}
\item{クラスターモデル}{}
{非周期条件の下で、固体側を有限の原子数のクラスターとして表現するモデル化。適宜水素終端化処理した上で、分子の吸着などを計算する。}
\item{クラスレート}{}
{結晶構造中に異分子が共有結合をすることなく内包されたもの。包摂化合物。メタン分子が氷状結晶中に内包されたメタンハイドレートなどが知られる。}
\item{クラスレートハイドレート}{}
{複数の水分子で作るかご型の構造中に気体分子が取り込まれた結晶。気体分子と水の混合物を加圧することにより生成する水和物}
\item{グラフェン}{}
{六角形二次元平面に周期的に配置された格子構造を持つ炭素結晶.}
\item{グラムシュミット直交化}{}
{直交化とは、いくつかの「線形独立だが互いに非直交なベクトル(または関数)の組」を、「互いに直交するベクトル(または関数)の組」に変換する操作を指す。直交化を施すことで、数学的表現が簡素になって取扱い易くなる。グラムシュミット直交化は、いくつか存在する直交化法の中でも概念的に最もシンプルなもの。}
\item{クリープ構成則}{}
{主として金属材料が高温状態にさらされた際に呈する非線形挙動を応力とひずみの関係として記述したもの。各種金属に固有の温度を超えると、荷重が一定でもひずみが時々刻々変化する、いわゆるクリープ変形が顕著になる。その挙動を応力ーひずみ関係として記述したもの。}
\item{クリロフ部分空間解法}{}
{連立一次方程式の解を求める際に使用される行列解法の一つ。行列積を直接計算する代わりにベクトルを利用した解法の総称で、ロシアの数学者にちなんで名づけられた。現在最も主流の行列解法であり、具体例としてBi-CGSTAB法、GMRES法などがある。}
\item{グルーオン}{}
{強い力を媒介する粒子。}
\item{グルーボール、ハイブリッド粒子}{}
{グルーオンが複数個結合した複合粒子がグルーボール。これにクォークもからむとハイブリッド粒子と呼ばれる。}
\item{グローバルビュー}{}
{通常並列計算機は複数の計算機から構成される複合システムであり、個々の計算機間は別個のビュー、すなわち実行の状態(メモリ)をもつ。グローバルビューは特別なソフトウェアもしくはハードウェアにより並列計算機全体で単一のビューを共有する方式であり、これによって並列計算機のプログラミングが大幅に簡略化される。}
\item{グローバルファイルシステム}{}
{並列計算機のすべての計算ノードから参照可能な共有ファイルシステム。一般に利用者の恒久的なファイル置き場として使われ、ローカルファイルシステムと比較して大容量かつ安定性を重視した構成となっている一方、読み書きの速度は限定的である。}
\section{け}
\item{形態学}{}
{細胞の形状と組織の広がりなどを調べる方法}
\item{ゲージ群}{}
{力学系の作用が余分な自由度をもち、その自由度に対して変数変換しても作用が不変な場合がある。このような変換をゲージ変換といい、これは一般に群をなす。これをゲージ群という。ゲージ変換のもとで不変な理論をゲージ理論と呼ぶ。この場合余分な自由度は観測にかからない。}
\item{ゲート}{}
{Hodgkin-Huxleyモデルの中で電流の開閉を司るスイッチの役割を果たす仮想概念}
\item{結合クラスター展開}{}
{無限次の摂動論に相当する電子相関理論。複雑なテンソル積和処理を伴う繰り返し計算が必要で、2次に比して精度は高まるが計算コストは高い。}
\item{原子核殻模型計算}{}
{原子核の構造を計算する手法の一つ。陽子と中性子の多体系である原子核を、適切な1粒子状態を基礎にして核力に忠実に、多体相関を含みつつ量子力学的に計算する。量子化学における配置換相互作用計算と類似した手法である。計算は大次元行列の固有値問題に帰着する。その解法としては行列の対角化に基づく従来型の方法と、重要な多体状態の基底を探す方法の2種類がある。}
\item{原子軌道基底}{}
{分子軌道を表現するための関数群。原子軌道を表す関数の線形結合で分子軌道を表現。}
\item{元素の起源}{}
{現在の宇宙の元素の組成は、ほぼ水素とヘリウムで構成されおり、そのほかの元素の量は無視できるほどである。宇宙誕生後の物質進化の過程を追うことで、さまざまな元素の組成比を理解することが元素の起源を探ることである。}
\section{こ}
\item{格子QCD(格子量子色力学)}{}
{QCDはクォークとグルーオンの強い力の力学であるが、解析的に解くことはできていない。数値的にQCDを取り扱うことができるように、4次元時空を格子に差分化した理論が格子QCDである。100TFlopsクラスの計算機が登場した2008~2009年に、クォークの複合粒子である陽子や中性子などの性質(質量やスピンなど)を計算で再現できるようになった。}
\item{格子気体法}{}
{流体問題を空間と流体の両方を離散化して解く計算手法}
\item{構造緩和}{}
{最初に仮定した物質の構造(=原子の配置)を原子に働く力が小さくなる方向に原子を動かすことでもっとも安定な構造に近づけること}
\item{構造多型}{}
{タンパク質などの巨大分子が複数の安定な構造を持つ性質}
\item{構造ゆらぎ}{}
{タンパク質分子が機能を発現させるためにその構造を変化させること}
\item{拘束付平均場}{}
{ある量が決まった値になるように条件を付けながら計算をする平均場理論。}
\item{高立体選択的合成反応}{}
{複数の立体異性体(配位子の付き方が立体的に異なる分子)の生成が考えられる化学反応において、触媒の利用などにより特定の立体異性体を選択的に多く作り出す反応のこと。}
\item{呼吸鎖}{}
{細胞の呼吸(ATPの生成)に関わるタンパク質群}
\item{骨格振動}{}
{2重結合や芳香環などの分子構造に起因する特徴的な振動。赤外やラマンで分光測定することにより、対象分子系の分子構造を推定できる。}
\item{混雑物}{}
{分子混雑環境において溶存するタンパク質、DNA、RNA、糖をはじめとする様々な分子}
\item{コンダクタンス}{}
{電気伝導度。すなわち抵抗の逆数}
\section{さ}
\item{再帰現象}{}
{相互作用する多数の粒子の運動において以前と同じ状態が準周期的に現れる現象}
\item{細胞環境}{}
{細胞内分子にとっての環境。分子が溶液中にあるときの環境と異なり、多くの分子で混み合っている。}
\item{材料強度発現機構}{}
{材料の破壊を発生・進行させるメカニズム。その破壊挙動は、主に材料内の力学場と材料固有の強度との相関により決定される。}
\item{サブボリューム}{}
{並列計算において,1プロセッサが担当する部分領域.なお,シミュレーションセルを空間分割して個々の並列プロセッサに割り当てる手法を領域分割法と呼ぶ.}
\item{差分法}{}
{微分方程式を数値的に解く際に用いられる離散化手法のひとつ。ある関数が2つの変数値に対してとる値の差を差分といい、この差分を変数値の差で割って得られる商を差分商と言い、この差分商を用いてもとの微分の近似値とすることで偏微分方程式の離散化を実現する。}
\item{残基}{}
{タンパク質、核酸、多糖類などの重合体を構成している単量体}
\item{参照曲率}{}
{計算要素内で形状を表現する時に用いるパラメータの一つ。形状の曲率の事。}
\item{三相界面}{}
{燃料極と固体電解質,空気極の三相が接する境界面.その長さが燃料電池の反応性を左右する.}
\item{散乱・束縛状態}{}
{2粒子以上の系において、各々の粒子の運動が有限の範囲に限定されるものを束縛状態、無限遠方まで許されるものを散乱状態という。}
\section{し}
\item{磁気回転不安定}{}
{差動回転(天体の各部分で異なる角速度を持つ回転)する磁気流体に起こる不安定性。通常の天体では内側の物質ほど角速度が大きい。 内側の物質は角速度が大きいため、外側の物質に先行する。しかし、磁場を通して内側の物質と外側の物質はお互いを引っ張りあう。すると内側の物質は一旦減速し、外側の物質は一旦加速する。内側の物質は減速すると、天体の重力に引っ張られてさらに内側に落下する。内側ほど角速度は大きいため、内側の物質は結局減速前よりも大きい角速度を持つことになる。外側の物質はこれとは逆に加速前よりも小さい角速度を持つことになる。すなわちこの不安定は、内側と外側の物質の角速度差がどんどん大きくなる不安定である。}
\item{自己相関時間}{}
{系を時間発展させてサンプリングする際、ある時刻でのサンプルと、それとは独立と考えられる次のサンプルを採取するまでに要する時間。}
\item{自己無撞着}{}
{セルフコンシステント(self-consistent)。}
\item{システムインパッケージ}{}
{英語でsystem in a packageのことで、1つのPackageの中に複数の半導体チップを集積することにより、システムレベルの高度な機能を実現して、実装密度の向上とコストダウンを実現する技術。}
\item{システム生物学}{}
{生命現象をシステムとして理解することを目指す学問分野}
\item{次世代シークエンサー}{}
{DNAを100塩基程度と非常に短く断片化し、それを並列に処理することにより高速に読み取ることのできる装置。読み込んだDNAは断片であるため部位の特定のため計算機を用いた参照配列との照合に多量の計算が必要である。}
\item{質量異常次元}{}
{エネルギースケールの変化に対する質量の振る舞いを記述し、相互作用による効果を表す。}
\item{シナプス遅延}{}
{シナプス前末端でカルシウム濃度が閾値を超えてからシナプス後膜でシナプス後電位が発生するまでの遅延}
\item{自発的対称性の破れ}{}
{系が本来持つ対称性の一部が自ずと破れて、より対称性の低い状態に系全体として落ち込むこと。この概念は相転移と密接に関連しており、たとえば水(液体)から氷(固体)への変化は水分子の併進対称性が失われることとして理解される。}
\item{シフト型通信}{}
{各プロセスが隣接する他プロセスに対して、一斉に一定方向のデータ送信をする通信形態をいう。}
\item{シミュレーションセル}{}
{シミュレーションの中で考慮する空間領域}
\item{重イオン}{}
{陽子、ヘリウムなどの軽い原子核を除く、重い原子核のことを指す。電子をはぎ取った原子なのでイオンと呼ぶ。}
\item{重合脱重合化}{}
{同種の分子が結合してより大きな構造を取ったり結合を解くこと}
\item{重陽子}{}
{陽子と中性子の束縛状態。二重水素の原子核。}
\item{重力の量子化(量子重力)}{}
{素粒子標準理論の中の相互作用を記述する部分は、量子力学の原理に則り量子化され、ミクロな世界での物理を矛盾なく記述できている。一方で、重力理論であるアインシュタインの一般相対性理論を量子力学の原理に則り量子化しようとすると、うまくいかない。一般相対性理論や何らかの重力の理論を量子力学と矛盾なく量子化すること。宇宙そのものの誕生時を理解するためには、量子力学が必要なミクロな世界での重力を理解する必要があるため、重力の量子化は理論物理学の長年の夢であるがまだ実現していない。超弦理論がその候補とされている。}
\item{主殻}{}
{調和振動子ポテンシャルによる一粒子軌道によって空間を展開した際に、縮退した一粒子軌道の集合を指す。}
\item{準粒子}{}
{相互作用している多体系を、近似的に自由に運動するある種の「粒子」の集まりとして記述することができるとき、この「粒子」を準粒子とよぶ。}
\item{状態空間モデル}{}
{時系列観測データのモデル化の方法の一つでデータを状態モデルと観測モデルに分離し記述する}
\item{状態方程式}{}
{物質の温度、圧力、エネルギー、密度、体積などの間に成り立つ関係式。}
\item{ショットガン法}{}
{ゲノムDNAを断片化し読み取りそれを計算機を用いてつなぎ合わせることにより染色体の連続したDNAを読み取る方法}
\item{真空偏極}{}
{真空における粒子・反粒子の対生成・対消滅過程。}
\item{神経成長因子}{}
{特定の細胞の神経細胞への分化を促進する因子となる分子}
\item{震源過程}{}
{地震は、発生源で断層が破壊されることによって生じる。この断層の破壊過程を、震源過程という。}
\item{信号情報処理のマルコフ過程}{}
{一個一個のイオンチャネルの挙動やレセプタとリガンドの結合はリガン度濃度や電圧などに対して確率的に挙動する}
\section{す}
\item{水平乱流}{}
{流れが乱れた状態(流体の粘性力に対して流れの慣性力が大きい状態)を乱流と呼ぶ。二次元(水平)乱流とは、大気のように成層が強い場で、鉛直方向の運動が制限され、水平方向の運動が卓越する状態を指す。水平乱流場においては、物質は水平方向に拡散される。}
\item{数値求積法}{}
{非解析的、近似的に積分値を求める手法。ガウス求積などの積分区間を区切る手法や、乱数を用いるモンテカルロ積分などがある。方法によって求積点数と誤差の関係が異なる。}
\item{スーパーBファクトリー}{}
{電子と陽電子を高頻度で衝突させることによってボトムクォークを含むハドロンを大量に生成し、その崩壊を詳細に調べることを目的とした加速器。従来のBファクトリーの数十倍のルミノシティを目指す。}
\item{スーパーセル}{}
{結晶中にとる事のできる周期セルのうち、基本セルよりも大きい物。基本セルよりも大きな空間スケールの構造揺らぎの表現に用いる。}
\item{スカイライン形式}{}
{疎行列に対するメモリ格納形式の一つで、バンド形式をより精緻化し、境界の輪郭線を行単位で正確になぞるようにしたもの。}
\item{スケール間相互作用}{}
{気象や気候現象に存在する複数の様々な時空間スケール(例えば、全球スケールや温帯高低気圧のスケールなど)の現象が相互に影響を及ぼしあっていること。}
\item{スタッガード型}{}
{格子上で定義されたクォーク作用の一つ。}
\item{ステップスケーリング}{}
{エネルギースケールをs倍(典型的にはs=2)ずつ不連続に変化させながら、結合定数などのエネルギー依存性を調べる数値計算手法。}
\item{ストークス力学}{}
{流れの状態を示すレイノルズ数が小さな場合に、流れを近似方程式で示す事が出来、これをストークス方程式と呼ぶ。近似方程式では非線形項である対流項を無視している。}
\item{ストレンジクオーク}{}
{標準模型に含まれる素粒子には6つの質量の異なるクォークがある。粒子質量の軽い順からアップ、ダウン、ストレンジ、チャーム、ボトム、トップと名前が付けられている。標準模型では質量以外の性質は同じである。ストレンジクォークは3番目に軽いクォークである。}
\item{ストレンジネス}{}
{ストレンジクォークが関与する量子数。正確には、ストレンジクォークの数とその反粒子の数の差。}
\item{ストロング・スケール}{}
{並列化の指標。計算量とcpu数が両方増えて行く時の計算効率。}
\item{スパイク列}{}
{複数の活動電位が連続して出る様}
\item{スピン液体}{}
{量子力学的な揺らぎや幾何学的フラストレーションの効果により、磁気モーメント間の集団的な秩序化が絶対零度まで妨げられた状態。}
\item{スピントロニクス}{}
{エレクトロニクスが物質中の電子が持つ電荷自由度だけを利用していたのに対し、スピン自由度も工学的に応用する技術。}
\item{スペクトル法}{}
{物理現象を表す偏微分方程式の時間積分法の一つで、物理変数の時間変化を直接計算するのではなく、周波数空間に置き換えて計算する手法。一般に高精度な解が得られるため基礎的な物理計算によく用いられるが、複雑な問題には対応が難しいとされている。}
\section{せ}
\item{正準化変換}{}
{Hartree-Fock方程式を解く際に非直交基底関数の組を変換して規格直交系を作る手法のひとつ。}
\item{静的縮約}{}
{連立一次方程式において、自由度の一部を削除することで、係数行列のサイズを縮小する方法。スタティック・コンデンセーション。}
\item{世界線表示}{}
{量子力学に従う系は空間次元に加えてもう一つ虚時間と呼ばれる軸を導入することで、計算機が扱いやすい複素数での計算が可能となる。その際、系の状態が虚時間方向でどのように発展するかをグラフ的に表現することを世界線表示と呼ぶ。}
\item{積分発火モデル}{}
{細胞外に抵抗と容量で接続された点として考え、シナプス後電流が複数の別の入力細胞からはいったとき、その時空間的統合としての細胞電位が閾値を超えたときに活動電位が起こり、結果過分極側に一定量電位がシフトすると考えるモデル。英語はIntegrated-and-Fire model}
\item{零点振動}{}
{量子力学的に絶対零度でも不可避の量子の振動}
\item{$0+$状態}{}
{原子核の基底状態や励起状態は、角運動量$J$とパリティ$\pi$で識別することができる。$0+$状態とは、$J=0$でパリティが$\pi=+$の状態。}
\item{線型応答理論}{}
{熱平衡状態にある系に、磁場や電場などの外場が加わった時、その外場による系の状態の変化(応答)を扱う理論。}
\section{そ}
\item{相対論的流体}{}
{相対性理論の枠組で扱う必要がある流体。速度が光速近くに達する流体や、中性子星のよう強重力場中の流体などがこれに対応する。}
\item{相変態}{}
{ここでは,固体電解質材における結晶構造の変化.}
\item{阻害活性}{}
{化合物が標的タンパク質の機能を阻害する性質}
\item{素過程}{}
{複雑な自然現象は、様々な物理(電磁気学, 熱力学, 流体力学等)が絡み合って生じている。しかし、少くない現象においては、関わる物理をいくつかの構成要素に分割し、その要素間の相互作用として記述することが可能である。そのような構成要素のうち、特に基本的な物理で比較的単純に数学的に表現することができるものを素過程という。たとえば、流体力学で支配される移流(力学)過程、放射伝達方程式で支配される放射過程などがそのような素過程である。}
\item{粗視化分子動力学法}{}
{複数原子からなる集団を一粒子とみなしその群としての運動をシミュレートする手法.計算量の減少を通じて大規模で長時間の分子シミュレーションを可能とする.}
\item{粗視化モデル}{}
{原子のグループをまとめて、一つの相互作用点として表し、相互作用数を大幅に減らしたモデル。たとえば、タンパク質のアミノ酸を一つの相互作用点を近似する粗視化モデルなどがある。}
\item{塑性加工解析}{}
{金属部品の成型プロセスにおける材料加工処理のシミュレーション。この際に大変形弾塑性解析を行う必要がある。}
\item{袖領域}{}
{差分法等のステンシル計算では隣接する要素、格子上のデータを参照する。このため、計算領域を分割して並列処理を行う際に、隣接ノードの境界データを保持する。この境界データを袖領域という。}
\item{ソフトウェアパイプライニング機能}{}
{コンパイラの最適化機能の一つ。ループ内で繰り返される一連のCPUの処理命令を1サイクルに1つずつ実行するのではなく、複数の処理命令を並列実行することで処理速度を向上させる。}
\item{ソリッド要素}{}
{構造解析において、連続体をそのまま表現するための有限要素。形状としては、四面体あるいは六面体などの形を有する。これとは別に、梁やシェルなどを表現するための構造要素がある。}
\item{素粒子標準理論(または素粒子標準模型)}{}
{自然界の物質を構成する素粒子の運動と、素粒子間の相互作用を記述する法則をまとめた理論。素粒子としては、クォークと呼ばれる6種類のスピン1/2のフェルミオンと、レプトンと呼ばれる6種類のスピン1/2のフェルミオンが含まれる。相互作用は電磁気力、弱い力、強い力の3つの相互作用を媒介する4種類のスピン1を持つボソンが含まれている。電磁気力と弱い力を分化させ、素粒子に質量を与えるヒッグス粒子と呼ばれるスピン0のボソンを含む。量子力学と矛盾しないように作られている。実験との比較でしか決まらない18(+$\alpha$)個の独立パラメータが含まれる。重力はここには含まれない。}
\item{素粒子標準理論に内在するエネルギースケールの階層性}{}
{素粒子標準理論のパラメータには以下のような階層性がある。 (1)弱い力の媒介粒子の質量が重い。(陽子の約80倍と約90倍) クォークやレプトンの質量はバラバラであるが、 (2)トップクォークは特に重い。(陽子の約170倍) (3)ニュートリノの質量が直接測定不可能なぐらい軽い。 質量はエネルギーと等価であるので、エネルギースケールの階層性という。このような階層性の起源を明らかにすることは現在の素粒子物理学の課題である。}
\section{た}
\item{ダークマター}{}
{暗黒物質とも呼ばれる電磁気力と強い力が作用しない仮説上の物質。電磁相互作用しないので、地上実験や天文観測では直接検出できない。ダークマターはエネルギーを持ち重力に影響を及ぼすことから、ダークマターによる重力レンズ効果や、銀河の回転運動の検証などで間接的にその存在が推定されている。シミュレーションにより、宇宙の大規模構造の生成にも重要な役割をしていることが分かっている。近年のWMAP(Wilkinson Microwave Anisotropy Probe)衛星による観測から、ダークマターは宇宙全体のエネルギーの内、約20%を占めていると考えられている。素粒子標準理論にはダークマターに該当する粒子はない。}
\item{第0近似的}{}
{実際の現象を細部まで捉えられてはいないが本質は捉えられている様子をいう。}
\item{大域構造}{}
{原子同士が直接触れ合うような短い距離でみられる構造ではなく、多数の原子の集団同士の関係が作り出す長い距離で特徴づけられる物質の構造のこと。}
\item{第一原理計算}{}
{電子シュレディンガー方程式を(半)経験的パラメータによる積分の近似を用いないで数値的に解く計算手法。化学では、非経験的計算とも呼ばれる。}
\item{第一原理ダウンフォールディング法}{}
{第一原理計算を用いて対象とする物質の個性を残しつつ注目するエネルギースケールに応じた有効模型を構築すること。得られた模型をより精緻な計算手法で解析することで非経験的かつ高精度な物性値の計算が可能となる。}
\item{大規模連立線形方程式}{}
{ここでは変数の数が数千万から数十億程度の連立線形方程式を想定している。}
\item{対称正定値}{}
{行列が対称かつ、その固有値がすべて正値であること。この性質を有する行列はより効率的に扱うことができる。}
\item{大振幅集団運動}{}
{多数の核子が一斉にある秩序を持って運動することを集団運動とよぶが、特にその運動の振幅が大きく、物理学で良く使われる調和近似などが適用できない集団運動を大振幅集団運動とよぶ。}
\item{対超流動}{}
{2つのボーズ粒子のペアからなるボーズ粒子の示す超流動現象。}
\item{大統一理論}{}
{自然界の4つの基本的な力である電磁力・弱い力・強い力・重力のうち、電磁力と弱い力の統合(電弱統一理論)に加えて強い力をも統合する理論。}
\item{ダイナミカル行列}{}
{結晶内の原子の相互作用を記述した行列。}
\item{タイリング}{}
{計算機上で、大規模なデータを配列の添え字ごとに細かく区切り、小さな部分配列(=タイル)の集合として扱うこと。行列のような2次元のデータ配列をタイリングすると、四角形のタイルを敷き詰めたようなイメージになることから。}
\item{タイルドディスプレイ}{}
{高解像度の表示領域を確保するため,複数のモニタを並べて配置したデバイス.通常,クラスタシステムなどで動作する.}
\item{多参照理論}{}
{電子の波動関数を表すために、複数の電子配置の重ね合わせを用いる理論。分子の解離状態などでは、単一のSlater行列式では良い波動関数が表現できず、多参照理論が必要となる。}
\item{多次元効果}{}
{対称性(球対称や軸対称など)を仮定し次元を落としたシミュレーションでは現れない現象。例としては対流などがある。}
\item{多重格子法}{}
{ポアッソン方程式を格子で離散化して反復法で解くような場合には、基本的に格子サイズ程度の短い波長の誤差が効率良く減衰する一方で、長波長の誤差はなかなか減衰せず、これが反復回数増大の原因となる。多重格子法は、格子サイズの異なる複数の格子を用意し、各波長の誤差を一様に減衰させることで反復回数の増大をおさえる数値解法である。}
\item{脱閉じ込め臨界現象}{}
{相転移でありながら、ランダウが提唱し相転移の標準的な起源として知られる「自発的対称性の破れ」の範疇に入らず、実在すれば教科書を書き換える発見になるとして注目されている、新しいタイプの臨界現象。}
\item{弾塑性構成則}{}
{主として金属材料の挙動を、応力とひずみの関係から記述する際の関係式を指す。金属材料は変形初期の段階では応力とひずみに線形関係がある、いわゆる線形弾性体であり、ある限界を超えると非線形な塑性挙動を呈するようになる。その限界値と、非線形挙動を応力ーひずみ関係として記述したものである。}
\item{タンパク質の折れたたみ}{}
{タンパク質がある一定の立体構造をとる過程}
\section{ち}
\item{チェックポイントファイル}{}
{計算の途中の状態を保存するファイル。万一計算が計算機の故障で中断した場合、このファイルから計算を継続実行できる。}
\item{地磁気異常の縞模様}{}
{海洋底の地磁気を調べて標準より強く帯磁している所を黒く塗ると海嶺と平行な縞模様が海嶺から両側に全く対称的に現れる。この縞模様は、海洋底が海嶺から湧き出して冷却する時に記憶する地球磁場が、その当時の地球磁場を反映して反転を繰り返しているためと説明される。}
\item{チャネルロドプシン}{}
{緑藻植物のクラミドモナスなどがもつ色素たんぱく質で、光が当たるとイオンを透過する。}
\item{中間子}{}
{パイ中間子やオメガ中間子などがある。1つのクォークと1つの反クォークが強い力で結合してできた粒子の総称。中間子にはいろいろな種類があるが、それらは2個のクォークの組合せによる違いや内部の状態の違いで理解されている。}
\item{中間子、パイ中間子、K中間子}{}
{ハドロンのうち、クォーク2個(クォーク・反クォーク対)からなるものが中間子(meson)。核力を媒介する粒子として湯川によって予言されたものがパイ中間子。sクォークを含む中間子の1つがK中間子。}
\item{中性子過剰核}{}
{陽子数に比べて過剰に多い中性子を含む原子核。不安定であり、安定な原子核になるまで中性子から陽子への$\beta$崩壊を繰り返す。}
\item{超新星爆発}{}
{大質量恒星の進化(一生)の最後に起こる爆発的現象。太陽質量の10倍より重い質量の恒星は、熱核融合反応により恒星の中心部に鉄の芯が形成される。鉄は熱核融合を起こさないため重力による収縮が起こり、鉄コアの温度が上昇していく。ある温度で鉄原子核はヘリウムや核子に分解する吸熱反応を起こし、恒星外層部の物質が中心に向かって急速に落下(重力崩壊)し中性子の芯が形成される。外層部からさらに物質が中性子の芯へ落下してきて中性子の芯に跳ね返され衝撃波が生じる。この衝撃波が恒星外層部を吹き飛ばし、超新星爆発を引き起こすと考えられていた。しかし、これまでの計算機シミュレーションでは、この機構によって爆発をうまく再現できていない。爆発機構の解明は重要な課題である。}
\item{超対称性}{}
{ボゾンは整数スピンを持つ粒子であり、フェルミオンは半整数スピンを持つ粒子である。それらを入れ替えるような操作を超対称性変換と呼び、その変換に対して理論が不変であるとき、その理論は超対称性を持つという。}
\item{超対称性粒子}{}
{超対称性理論では、標準理論に登場するすべての素粒子に対してペアとなる超対称性粒子を置く。標準理論のボーズ粒子に対してはフェルミ粒子、逆に標準理論のフェルミ粒子に対してはボーズ粒子が追加される。}
\item{超流動核}{}
{核子(陽子・中性子)がクーパー対を作ることでボーズ凝縮し、低温の液体ヘリウムのように超流動性を示す原子核。低温の金属における超伝導と類似した現象。}
\item{超流動固体状態}{}
{固体秩序と超流動秩序が共存した状態}
\item{調和振動子}{}
{$k x^2$のポテンシャルの中で運動する振り子、または量子。}
\section{つ}
\item{通信マスク手法}{}
{非同期通信や通信用スレッドの実装によって演算処理の背後で通信処理を同時に実行する手法。}
\item{強い力}{}
{素粒子標準理論では、すべてのクォーク間に平等に働く力。電磁気力に比べ100倍強い。クォークの間でグルーオンと呼ばれるボソンが交換されることで力が作用しあうと考える。強い力では、3種類のクォークを強固に一つにまとめる場合と、1種類のクォークと1種類の反クォークを強固に一つにまとめる場合がある。陽子や中性子は3種類のクォークからなる複合粒子であり、パイ中間子はクォークと反クォークからなる複合粒子である。}
\section{て}
\item{低次元構造体}{}
{一次元もしくは二次元の周期的な結晶格子構造を持つ原子構造体.一次元の例としてはナノワイヤー,二次元の例としてはグラフェン等のナノシートが挙げられる.}
\item{低侵襲治療}{}
{手術などに伴う痛み、発熱、出血などをできるだけ少なくする医療}
\item{低レイテンシ}{}
{通信の際に、データ転送などを要求してから、実際に送られてくるまでの遅延時間のことをレイテンシ(遅延)と呼ぶが、その遅延時間が短いこと}
\item{データ転置}{}
{多次元データの並列処理において並列化軸を切替える際に発生するデータ転送処理。}
\item{テクニカラー理論}{}
{標準理論を超えたモデルの一つで、ヒッグス粒子を複合粒子として考える。このモデルが妥当であるためにはQCDに似た性質を持ちつつも相互作用の強さの性質がQCDと違った特徴を持つ必要がある。}
\item{テクニ中間子}{}
{テクニカラー理論において予言される複合粒子の一種。}
\item{鉄よりも重い重元素の起源}{}
{恒星内部での熱核融合反応では、水素から始まる核融合反応は発熱反応であり、水素よりも安定な重い元素を合成する方向に進む。十分重い恒星では重力収縮と熱核融合反応の連鎖により、鉄原子核でできた恒星芯が形成される。しかし、鉄原子核まで合成が進むと鉄は最も安定な原子核であるので、発熱反応が終わり恒星芯での熱核融合反応は終了する。鉄よりも重い元素は恒星内部での熱核融合反応による元素合成では生成されず、中性子捕獲反応で生成されたと考えられる。超新星爆発は金やプラチナなどの、鉄よりも重い重元素の起源の一つと考えられている。}
\item{転位動力学法}{}
{結晶中の線状欠陥である転位の運動をシミュレートする手法.塑性を支配する,結晶すべり挙動の解析に用いられる.転位に働く力をモデル化することで,古典分子動力学法に比して大規模かつ長時間のシミュレーションが可能となる.}
\item{展開係数}{}
{ある関数を基底関数の線形結合で表した際の、それぞれの基底関数のもつ重み。}
\item{電荷移動型ポテンシャル}{}
{分子動力学法で用いる,電荷の局所的な移動を考慮した原子間ポテンシャル.原子における電荷の偏りを取り入れることで,電気陰性度の異なる異種原子間の結合を高精度に表現できる.}
\item{電気生理}{}
{細胞の電位を測定することによって生理的な性質を知ろうとする体系}
\item{点欠陥}{}
{結晶中の不純物または空孔。}
\item{電源グラウンドバウンスノイズ}{}
{LSI内部の回路動作に伴う電流の時間変化に起因して発生するLSI内の電源及びグラウンド配線部分の電圧ノイズ。}
\item{電子捕獲}{}
{原子核が電子を捕獲することで、陽子を中性子に変換させる反応。ニュートリノが放出される。}
\item{電磁気力}{}
{電気力や磁気力による相互作用の総称。素粒子標準理論では、電荷を持つ粒子の間で光子(フォトン)が交換されることで力が作用しあうと考える。}
\item{転写因子}{}
{DNAに特異的に結合し、DNAの遺伝情報のRNAへの転写を促進、あるいは逆に抑制するタンパク質}
\item{テンソル縮約}{}
{二つのテンソル量を掛け合わせて、新しいテンソル量を得る操作。行列どうしの掛け算も、テンソル縮約の一つである。}
\item{テンソル力、3体力}{}
{核子(陽子・中性子)間に働く相互作用(核力)は、核子間の相対距離だけに依存する中心力とそれ以外の非中心力に分類できる。非中心力の代表がテンソル力(tensor force)。2核子のスピンの向きと配置の向きに依存する。これら2核子間の相互作用に加えて、原子核の定量的な記述には、3核子に働く3体力が不可欠であることが知られている。}
\item{天体降着流・噴出流}{}
{天体に向って落下する物質の流れと、天体から放出される物質の流れ。一般に中心天体に向かって円盤状に降着し、回転軸方向にビーム状に放出される。}
\item{転置転送}{}
{行列の転置操作に用いられるデータ転送パターン。}
\section{と}
\item{同時スイッチングノイズ}{}
{LSIの複数の外部出力信号がハイレベルからローレベルまたはその逆方向にほぼ同一のタイミングで変化する時にLSI内部の出力回路用の電源及びグラウンド配線に発生する電圧ノイズ。}
\item{トーラス状}{}
{ドーナツ型の幾何形状。磁場閉じ込め核融合炉ではプラズマを磁力線で覆い、かつ、端や磁場のゼロ点をもたないトーラス状の磁場を用いて高温の燃料プラズマを保持する。}
\item{トカマク装置}{}
{磁場閉じ込め核融合炉において最も有望な方式の一つ。トーラス状の磁場を発生するのに、トーラスに沿って並べたコイルとプラズマ中の電流を用いる。プラズマ中の電流を電磁誘導で駆動する場合にはパルス運転となるが、中性粒子ビーム等が誘起する電流を用いる定常運転も提案されている。トーラス断面形状が一様となる特徴(軸対称性)があり、核融合反応で発生する高エネルギー$\alpha$粒子の閉じ込めに優れる。}
\item{トポロジー励起}{}
{複数の縮退した量子多体系の基底状態が局所的な情報だけからは互いに区別がつかず、巻きつき数などのトポロジカルな量によってのみ区別できる場合を、トポロジカルな状態という。また、このトポロジカルな量を変えるような非局所的な励起をトポロジー励起と呼ぶ。}
\item{トラジェクトリスナップショット構造}{}
{分子動力学シミュレーションでの履歴(トラジェクトリ)中の1構造のこと}
\item{ドラッグデリバリーシステム}{}
{必要な薬物を必要な時間に必要な部位へと作用させるために、薬物の体内分布を制御し、患部に薬剤を届ける仕組み(Drug Delivery System: DDS)}
\item{トレーサ}{}
{一般的に大気や海洋の流れの影響を受けて、モデル内を移動する微量気体成分、溶存成分や個体のことを(パッシブ)トレーサーと呼ぶ。海洋生態系モデルにおいては栄養塩、植物・動物プランクトンや魚類がトレーサにあたる。}
\item{トンネル効果}{}
{量子がそのエネルギーより高いポテンシャルの山を越える、もしくはトンネルを抜けるようにくぐること。}
\section{に}
\item{二次高調波}{}
{入射電磁波と相互作用する物質中において、非線形光学効果により発生する入射電磁波の2倍の周波数を持つ電磁波のこと。}
\item{2次摂動論}{}
{HF計算では考慮されない電子相関(平均場からのずれ)を摂動論に従って取り込むPost-HF法の中で、2電子励起だけを考慮する基本的なアプローチ。強相関系には適用できない。}
\item{2電子クーロン反発積分}{}
{電子の振る舞いを調べるにはシュレディンガー方程式を解く必要があるが、その中で2個以上の電子を扱うためには電子間の反発を表すクーロン項を取り扱う必要がある。数値的計算では、2つの電子間の反発を積分表現を用いて表す。このときに必要となる積分を2電子クーロン反発積分と呼ぶ。量子化学計算では、2電子クーロン反発積分は数値計算の律速となるために、その取り扱いが重要となっている。}
\item{ニュートリノ}{}
{素粒子でレプトン族の一種。電子ニュートリノ、ミューニュートリノ、タウニュートリノの3種類が確認されている。電磁気力を受けない中性粒子で、弱い力と重力が作用する。このため検出は難しい。太陽中心付近での熱核融合反応で電子ニュートリノが発生し、地球には1平方センチ当たり毎秒660億個やってきているが、ほぼ地球をすり抜ける。近年、質量がゼロではないことが分かったが、ほぼゼロであり詳しい質量は不明である。超新星爆発では中心部に中性子の芯が形成される際に大量にニュートリノが発生し外部に放出される。1987年に大マゼラン星雲で超新星爆発(SN1987A)が起こり、超新星爆発由来のニュートリノが初めて地球上で観測された。日本の「カミオカンデ」ニュートリノ観測施設では詳細な観測が行われ、超新星爆発機構の理解に貢献した。}
\item{ニューロインフォマティックス}{}
{神経データベースや情報理論のような情報学的な手法を神経科学で手用させる学問領域}
\section{ぬ}
\item{ヌクレオソーム}{}
{真核生物の核におけるDNAとタンパク質の複合体であるクロマチンの構成単位}
\section{ね}
\item{熱揺らぎ}{}
{原子・分子程度の微小な粒子の熱運動に由来する運動エネルギー程度のエネルギー}
\section{の}
\item{ノルム保存型擬ポテンシャル}{}
{結晶内の電子の波動関数を平面波で展開するために真のクーロン型ポテンシャルの代わりに用いられるのが擬ポテンシャル。そのうちカットオフ半径内の電価(ノルム)を変えないのがノルム保存型擬ポテンシャル。}
\item{ノンブロッキング通信}{}
{並列計算における通信方法の一つ。データの送受信を行う際に、送受信の完了を待たず、他の処理を開始する通信方法。}
\section{は}
\item{ハートリー項}{}
{2つ以上の電子が存在するときに、電子と電子との間にはクーロン的な反発する力が働くが、それに関係するポテンシャル(位置エネルギー)もしくはエネルギーのことを指す。}
\item{ハートリーフォック(HF)計算}{}
{電子間の反発を平均場近似の下で記述し、系の分子軌道を変分的に求める手法。}
\item{ハートリーポテンシャル}{}
{電子密度の空間分布で決まる静電ポテンシャル}
\item{バイアスポテンシャル}{}
{自然状態では滅多に起こらないが重要な化学反応を人工的に高い頻度でシミュレーション上発生させるために加える原子間ポテンシャルのこと。Metadynamics法はバイアスポテンシャルを生成する。}
\item{バイオインフォマティクス}{}
{生物学的な問題をハイスループットデータなどとアルゴリズムを組み合わせて計算機を用いて解決する研究手法}
\item{バイオミネラリゼーション}{}
{生物が結晶や無機鉱物を産生すること。骨や歯、貝殻などが身近な例。}
\item{バイオミメティック}{}
{生物が持つ優れた機能を人工の物質で実現しようとする化学}
\item{バイセクションネットワークバンド幅}{}
{通信網の性能の指標の一つ。通信網の中の計算ノードを2等分し、その分割された部分同士の間で単位時間あたりに通信できるデータ量の事。}
\item{バイナップ}{}
{立体選択的合成反応において広く利用されている配位子。バイナップ-ルテニウム触媒を用いた不斉水素化反応を開発した野依良治は2001年のノーベル化学賞を受賞した。}
\item{ハイパー核}{}
{ストレンジクォークを含むバリオンをハイペロンと呼ぶ。ハイパー核とはハイペロンを含む原子核の総称。}
\item{ハイブリッド汎関数}{}
{実験値をより良く再現するために、従来の汎関数にHF交換相互作用の要素を取り込んだ汎関数}
\item{ハイペロン}{}
{ストレンジ(s)クォークを含むバリオンはハイペロンと総称され、ラムダ粒子、シグマ粒子、オメガ粒子などがある。}
\item{バタフライ演算}{}
{高速フーリエ変換などにあらわれる演算および通信パターン。}
\item{発火}{}
{スパイク様の活動電位が発生する様}
\item{ハドロン}{}
{強い力で結びついたクォークの複合粒子の総称。ハドロンはクォーク3個からなるバリオンとクォーク・反クォーク対からなるメソンに分類される。陽子や中性子はバリオンの一種である。}
\item{ハドロン、バリオン、ハイペロン、ラムダ粒子、シグマ粒子、オメガ粒子}{}
{強い力で結びついたクォークの複合粒子の総称がハドロン。ハドロンの中で、クォーク3個からなるものがバリオン。陽子・中性子もバリオンの一種で2種類のクォーク(u,d)から構成されている。ストレンジ(s)クォークを含むバリオンはハイペロンと総称され、ラムダ粒子、シグマ粒子、オメガ粒子などがある。}
\item{ハドロン共鳴}{}
{強い相互作用により様々なハドロンが形成されるが、その多くは短時間で崩壊するため、共鳴状態と呼ばれる。}
\item{ハドロン行列要素}{}
{相互作用を記述する演算子をハドロン状態で挟んだ行列要素。}
\item{ハドロン相}{}
{クォークはハドロン中に閉じ込められており、単独では取り出すことができない状態。}
\item{ハミルトニアン}{}
{系の(量子)力学を表現するもの。直接には系の時間発展を記述する。}
\item{ハミルトニアン行列}{}
{ハミルトニアン(系のエネルギーを表す量子力学的演算子)を行列表現したもの。}
\item{パラメータスキャン}{}
{入力パラメータを変更して多数のシミュレーションを実行し、設計パラメータ等に対する性能や機能の依存性を検証すること。}
\item{バリオン}{}
{クォーク3つから成る粒子の総称。陽子や中性子はバリオンである。一方クォーク1つとと反クォーク1つから成る粒子はメソン(中間子)という。バリオンやメソンは強い相互作用をする粒子であり、バリオンとメソンを総称してハドロンという。}
\item{パリティ}{}
{空間の反転に対して系が対称性をもつときの量子数。}
\item{バルクナノメタル}{}
{一般の金属よりも小さな結晶粒からなる金属材料.強度,延性,靱性等の機械的性質の向上が見込まれる.}
\item{バルクひずみ}{}
{複数の材料が混ざった状態での計算要素内のトータルのひずみ量}
\item{バレンス粒子、バレンス空間}{}
{原子核の中の核子のうち、その構造の決定に特に重要な影響を与えるものをバレンス核子(粒子)と呼ぶ。近似的に、これらのバレンス粒子だけを取り扱った計算ができ、量子力学の計算ではバレンス粒子の運動を記述するヒルベルト空間を扱うため、これをバレンス空間と呼ぶ。}
\item{バンドギャップ}{}
{電子が占有出来ない禁止帯のこと。最高占有軌道準位と最低非占有軌道準位とのエネルギー差に対応する。}
\section{ひ}
\item{光格子}{}
{対向するレーザー光を用いてその中を運動する粒子に対して周期的なポテンシャルを作り出す。その結果、粒子は結晶格子点に閉じ込められた粒子のように振舞い、そのような系のことを光格子と呼ぶ。}
\item{光生理学}{}
{化学的あるいは遺伝子光学的な光センサ-分子や光による刺激素子の発展を背景に蛍光顕微鏡のような光学的な方法で生体の活動をしる学問体系}
\item{非線形光学応答核磁気共鳴}{}
{核スピン間の非線形相互作用に伴い生じる高調波を利用する核磁気共鳴法}
\item{非線形振動子系}{}
{複数の非線形振動子が結合されたシステム。非線形振動子とは、運動が初期値に比例しない振動子(ばねのように振動する要素)のこと。カオスや同期など、様々な興味深い現象を示すことが知られている。}
\item{光分解}{}
{原子核が光(ガンマ線)を吸収して分解する反応。}
\item{非局所擬ポテンシャル}{}
{内殻電子などの及ぼす影響をポテンシャルに置き換えたもののうち、位置以外の要素(角運動量など)に依存するもの。}
\item{歪速度テンソル}{}
{速度場の空間的な変化を表す速度勾配テンソルから、回転を表す反対称成分を除いた対称成分で、変形の速度を表す。}
\item{非静水圧}{}
{大気や海洋の支配方程式を考える際、水平方向に十分大きな現象(気象では数十キロ以上)に着目する場合は、重力と鉛直方向の気圧傾度力が釣り合っていると近似(静水圧近似または静水圧近似と呼ぶ)することができる。全球を対象とした多くの大気・海洋モデルでは、静水圧近似した方程式が用いられている。一方、より細かな現象に着目する場合などは、静水圧近似が成り立たず、鉛直方向の運動方程式を陽に考慮する必要がある。このような方程式を、非静水圧の方程式と呼ぶ。}
\item{非摂動ダイナミクス}{}
{摂動的手法では解析が難しく、その本質を理解するためには非摂動的手法を必要とする力学現象。}
\item{ヒッグス場}{}
{素粒子が質量を持つ仕組みを説明する理論であるヒッグス機構において導入されるスカラー場。}
\item{ヒッグス粒子}{}
{標準理論において電弱相互作用から弱い相互作用と電磁相互作用を分化させ、クォークやレプトンなどに質量を与える重要な役割を担っている。}
\item{ビッグバン}{}
{現在広く受け入れられている学説によれば、宇宙は約137億年前に大きな爆発(ビッグバン)のように膨張して現在に至ったとされる。}
\item{ビッグバン原子核合成}{}
{ビッグバン宇宙誕生直後に起こった原子核の合成を指す。宇宙誕生直後、宇宙全体は超高温高密度であった。宇宙誕生後ごく初期には物質はクォークの状態であったが、宇宙が膨張し冷えるとともにクォーク同士が結合し陽子や中性子を構成するようになる(宇宙開闢後約$10^{-6}$秒)。その後、温度が下がると、いくつかの陽子と中性子は結びつき、一部がヘリウム原子核などを形成する元素合成が始まる(宇宙開闢後約3分から約20分の間)。さらに温度が冷えると元素合成は終了し、軽原子核は安定な原子核に崩壊し元素比率が固定される。この過程では無視できる量のリチウム7までの元素と水素1とヘリウム4が生成される。宇宙開闢後約3分から20分の間で生成された原子核を理論的に計算することができ、現在の宇宙の元素質量の割合が(水素1が約75%、ヘリウム4が約25%)であることを説明する。一方でビッグバン原子核合成ではわれわれになじみ深い炭素、鉄、金、銀などのリチウムより重い元素は全く生成できない。}
\item{非熱的分布}{}
{粒子の速度分布が熱的でない分布。熱的である分布とは、粒子同士が(衝突などの)相互作用を繰り返すことで達成される正規分布(マクスウェル分布)のことである。非熱的分布は相互作用がない(または少ない)状況で存在しうる。非熱的分布の下では、熱的分布では存在しえない高速な粒子が存在することがある。}
\item{微物理過程}{}
{大気中で雲を構成する水滴・氷晶(雲粒)が、発生してから、雨・雪などの降水現象として地表面に落下する、もしくは蒸発により消滅するまでの一連の成長・消滅過程をさす。雲粒同士が大気中で衝突して併合する過程、雲粒が凍結・融解する過程などがある。}
\item{標準脳座標系}{}
{個体差を補償するように作られた脳内の標準座標系}
\item{ヒルベルト空間、模型空間}{}
{量子力学では、系の状態は抽象的なヒルベルト空間の中のベクトルに対応している。この空間は無限次元であるが、実際の数値計算ではこれを有限の大きさの次元、しかもなるべく小さい次元の空間にする必要がある。このようにして計算に適した形に抜き出された空間を模型空間と呼ぶ。}
\section{ふ}
\item{ファインマン振幅}{}
{量子力学に基づいて素粒子反応の確率を計算する場合、絶対値の2乗が反応確率となる不変散乱振幅というものを計算する。通常普遍散乱振幅を解析的に厳密に計算することは困難であるため、摂動理論を用いて近似的に計算していく。摂動論では次数ごとにファインマン図形に基づく計算を行なう。この様な摂動計算による不変散乱振幅をファインマン振幅という。これにより素粒子反応の散乱断面積(反応確率)を求めることができる。}
\item{フィラメントワインディング}{}
{主に炭素繊維強化複合材料製の高圧容器を作製する際に用いられる製法。炭素繊維を数万本束ねた炭素繊維束を、ライナーと呼ばれる内容器に巻き付けて成型する方法。炭素繊維強化複合材料製高圧容器は、燃料電池自動車用高圧水素容器として使用され、高信頼性と軽量化の両立が求められている。}
\item{フーリエモード展開}{}
{時間微分を含む偏微分方程式を、正弦波の重ね合わせであるフーリエ級数に変換すること。複雑な波動を単純な波の重ね合わせとして表現することができる。}
\item{フェムトスケール}{}
{ハドロンや原子核の大きさ程度のミクロな世界。fm(フェムトメートル)は$10^{-15}$m。}
\item{フェムト秒}{}
{1,000兆分の1秒が1フェムト秒。1フェムト秒は、光の速さ(秒速約30万キロメートル)でも0.3ミクロンしか進むことができないほどの極短時間。}
\item{フェルミオン}{}
{フェルミ粒子。スピン角運動量が半整数倍である。フェルミオンには、クォーク、電子、ニュートリノ、陽子、中性子などがある。}
\item{フォノン分散関係}{}
{結晶の格子振動を量子化したのがフォノン。そのエネルギと波数との関係が分散関係。}
\item{フォルトトレランス機構}{}
{1ノードが故障したとしても補完により計算が止まらない仕組み}
\item{フォンビルブランド因子}{}
{血中にある凝固因子のひとつ}
\item{ブシネスク近似}{}
{流体を非圧縮性とし、圧力の変化に伴う密度変化は無視するが、温度変化に伴う密度変化は考慮する近似手法。}
\item{不純物偏析}{}
{結晶中の不純物が表面や欠陥など何らかの構造の周辺に集まること。}
\item{不定性}{}
{ここの文脈では、実験データの不足により、相互作用(力)の方程式(またはその元となるポテンシャル)を、実験データから良く決めることができないこと。}
\item{部分空間対角化}{}
{占有電子軌道など注目している一体電子軌道を基底とした空間でハミルトニアンを表現しそれを対角化すること}
\item{フラグメント}{}
{フラグメント分子軌道法計算を行うために分子全体を部分系に分割した際の構成単位のこと。}
\item{フラグメント探索}{}
{化合物設計プロセスにおいて化合物の部品(フラグメント)を探し出すこと}
\item{プラットホームシミュレータ}{}
{その中で独自のスクリプト言語を持つことにより多数の異なった現象が扱われるようになったシミュレータ}
\item{フレーバー}{}
{クォークとレプトンの種類を表す。たとえば、クォークにはアップ、ダウンなどの種類があり、レプトンには電子、ミュー粒子などがある。}
\item{ブロッキング}{}
{配列のデータ処理をする際にデータ転送速度が高速なキャッシュやメモリに保持可能なデータサイズを考慮して配列を区分することで、処理性能の向上を図る性能チューニング手法。}
\item{分割統治法}{}
{分割統治法は大規模な問題を効率的に解くアルゴリズムの一つで、そのままでは解決することが難しい大きな問題をいくつかの小さな問題に分割して個別に解決していくことで最終的に大きな問題を解決する方式。量子化学計算のための分割統治法はWaitao Yang教授(現デューク大学)より考案された。}
\item{分子混雑環境}{}
{細胞内のように、タンパク質をはじめとする様々な分子が高密度で存在する込み合った環境。分子は溶媒中における孤立した環境下とは異なった性質を示す。}
\item{分子動力学シミュレーション}{}
{原子間力に基づき、運動方程式を数値的に解き、分子の運動をシミュレーションする計算方法}
\item{分子モーター}{}
{生体内でATPなどのエネルギーを機械的な動きに変換する分子}
\item{分子モデリング}{}
{分子の立体構造を、計算機中で構築すること}
\item{分数量子ホール効果}{}
{半導体のヘテロ接合面等において実現される2次元電子系に強い磁場をかけると、低温でホール抵抗の値が量子化される現象が起こる。この値は$e$を電子の素電荷、$h$をプランク定数とすると$(p/q)\cdot(e^2/h)$と表される。ここで、$p$と$q$は整数であり、$q$が3以上の奇数で$p/q$が整数とならない場合を分数量子ホール効果と呼ぶ。これは物質中において分数電荷を持つ新たな素励起が生じるために起こる現象であり、発見者のTsui, Stormer, Laughlinは1998年にノーベル物理学賞を受賞した。}
\section{へ}
\item{閉殻・開殻配位}{}
{希ガスの原子が特別に安定化するように、陽子・中性子数がある決まった数(魔法数)になると原子核も安定化する。このような原子核の核子は閉殻配位に対応すると称される。逆に陽子・中性子の数が魔法数からずれたものを開殻配位と呼ぶ。}
\item{ベイジアンネットワーク}{}
{統計的因果モデルの一つで因子間の因果関係を点と有向枝からなるネットワークで表現したもの.バイオインフォマティクスでは遺伝子発現制御ネットワークの推定・モデル化で用いられる}
\item{ベイジアンフィルタ}{}
{ベイズ統計に基づくデータの学習・分類法}
\item{平面波基底}{}
{分子軌道を表現するための関数群。平面波を表す関数の線形結合で分子軌道を表現。}
\item{ヘテロな構成のCPU}{}
{機能の異なるコアを組み合わせたCPU}
\item{ヘム}{}
{鉄イオンを含む化合物。しばしば、タンパク質に含まれ機能の発現に重要な寄与をする。}
\item{変形核}{}
{形状が球形からずれて変形した原子核。}
\section{ほ}
\item{ポイントベースレンダリング}{}
{点群を基本とした画像生成手法.並列処理に向いた方法で,画像の品質を点の数により調整できる.}
\item{ボーズ・アインシュタイン凝縮}{}
{多数のボース粒子が一つの量子状態を占めることで現れる物質の状態。}
\item{ボーズ系モット転移}{}
{ボーズ粒子が互いの間に働く斥力相互作用によって絶縁体化すること。}
\item{ホールセルプランプ}{}
{細胞にパッチ電極を接続して全体を電位固定できる状態にして測定する方法}
\item{ボクセルデータ}{}
{形に沿った線や面で形状を表現するのではなく、空間を直方体で分割しその直方体内部の分布情報で形状を表現する方法。二次元の映像を示すピクセル(Pixel)に対して三次元(Volume)を表すボクセル(Voxel)}
\item{ポストスケーリング時代}{}
{大規模集積回路は、スケーリング(比例縮小)にもとづきトランジスタの微細化により高性能化と高集積化を同時に実現してきたが、今後は発熱や消費電力により困難となると予想されている。スケーリングの限界以降(ポストスケーリング)では、全く新しい指導原理が必要とされる。}
\item{ボソン}{}
{ボース粒子。スピン角運動量が整数倍である。ボソンには、素粒子間の相互作用を媒介する粒子である、光子やグルーオンなどがある。}
\item{ボゾン系}{}
{構成粒子がボーズ粒子である量子系}
\item{ボリュームレンダリング}{}
{ボリュームデータに対する画像生成手法の一つ.データの内部構造や全体の様子を透過的なイメージで表現することができる.}
\item{ポロイダル・トロイダル展開}{}
{任意のソレノイダル場は、トロイダルポテンシャルとポロイダルポテンシャルの二つのスカラー場で表現された二つの項の和として一意に分解できる。}
\section{ま}
\item{マイクロカプセル}{}
{極小のカプセル内に薬剤等を内包した物}
\item{マイコプラズマ}{}
{真正細菌の一種でゲノムサイズが小さく、細胞サイズも小さい}
\item{マイナー・アクチナイド核}{}
{ウラン・トリウムに代表される重元素をアクチナイドと呼ぶが、自然界に存在する安定な(寿命が非常に長い)ものの他に、原子炉などでは寿命の短いアイソトープが作られており、これらをマイナー・アクチナイドと呼ぶ。}
\item{膜輸送体}{}
{生体膜を貫通し、膜を通して物質の輸送をするタンパク質の総称}
\item{魔法数}{}
{原子核は陽子と中性子から構成されているが、ある特定の数の陽子または中性子を含むとき原子核は特に安定となる。この数のことを魔法数と呼ぶ。古くからよく知られている魔法数として2(ヘリウム:4He)、8(酸素:16O)、20(カルシウム:40Ca)などがある。}
\item{マルコフ連鎖}{}
{離散的な時系列を生成するための確率過程の一種で、ある時刻での状態は直前の時刻での状態のみに依存して決まり、それ以前の履歴と無関係である性質(マルコフ性)を持つ。}
\item{マルチグリッド型前処理}{}
{連立一次方程式の解法を使用する際、行列の収束性を向上するために導入される前処理法の一種。疎・密の計算格子に対して順に解を求め、反復計算において早く収束解が得られるようにした手法。構造型の格子を直接用いる方法、代数的に疎格子を表現する方法など、種々存在する。}
\item{マルチコンパートメント}{}
{神経線維を多数のシリンダ-様のコンパートメントの連なりと考えるモデル}
\item{マルチスケール・マルチレゾリューション法}{}
{幅広い時空間にまたがる対象に対し、それぞれの階層・解像度での計算を連成させるシミュレーション法}
\item{マルチフェロイクス}{}
{強磁性と強誘電性など二つ以上の秩序状態が物質中に共存し、互いに関係を持つ状態。これにより磁場(電場)をかけることで誘電性(磁性)を制御すること等が可能となる。}
\section{み}
\item{ミセル}{}
{疎水基と親水基を併せ持つ界面活性剤分子が、溶媒中において球状や棒状に自発的に会合した分子集合体}
\item{密度行列くりこみ群}{}
{強相関系の数値的計算手法のひとつ。特に1次元、または2次元的な電子構造を持つ低次元強相関系の研究に用いられる。}
\item{密度汎関数(DFT)法}{}
{系の電子エネルギーが電子密度の汎関数で与えられるコーンシャム方程式に基づき固体系や凝集系の電子状態を計算する手法。汎関数のバリエーションは多数あるが、物理分野ではBLYPがよく用いられる。}
\item{ミューオン異常磁気能率}{}
{ミューオンはレプトン族のうち2番目に重い粒子。質量以外の性質は電子と同じ。質量は電子の約200倍である。ミューオンはスピン1/2で自転しているため小さな磁石となっている。磁石の強さを磁気能率(磁気モーメント)という。磁気能率は量子力学に基づく計算と量子力学を使わない計算で違いが生じるため、その差を異常磁気能率と呼ぶ。ミューオンの磁気能率は高精度(相対誤差約0.5×$10^{-6}$)で計測されている。素粒子標準理論を用いた理論計算が可能である。2012年現在、理論計算と実験値は相対的に約21×$10^{-6}$ずれている。ずれの原因は、理論計算に含まれる精度不足である可能性と新しい物理の兆候である可能性がある。}
\item{ミュー粒子}{}
{レプトンの一種。電子と同じ性質を持つが質量が異なる。}
\section{め}
\item{メソ降水系}{}
{水平スケールが100km程度(メソスケール)の積乱雲の集合体である。単純な集合体ではなく、上昇・下降流域といった構造を持つ「系」であるため、単一の積乱雲に比べて寿命が長い(6時間以上)。大気の状態や地域特性によって形態を変え、停滞すると同じ場所に多量の降水をもたらす。}
\item{メタゲノム}{}
{特定の環境中の微生物群など単一種毎のゲノム解析が難しい場合に,その生物群内全体のゲノムの集合をひとつのゲノムとしてとらえる考え方}
\item{メタマテリアル}{}
{自然界では見られない性質を示す人工的に作られた物質一般を指す言葉であるが、特に負の屈折率を持つ物質を指すことが多い。光の波長よりも小さな物質で特殊な高次構造を作ることによって実現できる。その極めて特殊な光学的性質を利用した応用科学的研究も盛んに行われている。}
\item{メッシュ/トーラス}{}
{計算ノード間の通信ネットワークの形態の一つ。多次元の格子状のもの。格子の端を周期的に結合した物はトーラスと言う。}
\item{メモリバランス型}{}
{エクサスケールシステム構成例の一つ。 演算性能100PFLOPS/ メモリ帯域100PB/s メモリ量100PBがめどの構成}
\section{も}
\item{モデル脊椎動物}{}
{線虫(神経数300)・ショウジョウバエ・カイコ等の昆虫(神経数10万)は遺伝子が同定され、ある程度生理実験も可能な無脊椎系のモデル生物であるが、同様な意味で、脊椎動物においては・ゼブラフィッシュ(神経数100万)・マウス(神経数1億)などが世代が短く遺伝子が同定されており、かつ生理実験も可能な比較的単純なシステムを持つモデル脊椎生物といえる。}
\section{ゆ}
\item{ユークリッド時空}{}
{ユークリッド幾何学が成り立つ4次元時空。時間方向と空間方向の区別はない。}
\item{有限温度}{}
{非ゼロの温度を持つ物理系。}
\item{有限格子間隔効果}{}
{格子間隔が有限であることから生じる系統誤差}
\item{有限フェルミ多体系}{}
{核子(陽子と中性子)はフェルミ粒子であり、地球上に存在する原子核は最大でも数百個の核子から成っている。無限に近い粒子数の多体系と区別するため、有限多体系とよび、数値的にも多くの特有の困難がある。}
\item{有限密度}{}
{非ゼロの密度を持つ物理系。}
\item{有効媒質法}{}
{溶液分子の周りの溶媒の分布確率を表す分布関数を求める理論。様々な種類の分布関数理論があるが,特に3D-RISM法はタンパク質やナノチューブといった大きな分子の溶媒和を扱うことができる。}
\item{有効模型的アプローチ}{}
{特異性が強い核力を直接扱わずに、数値的に扱いやすい核力(有効相互作用)に変換する方法。上記の「カイラル有効場理論」と似た概念で、特定のエネルギー領域、制限されたヒルベルト空間における原子核多体問題で用いられる。}
\item{溶媒和エネルギー}{}
{孤立状態の溶質分子が溶媒中へと移行することに伴って変化する自由エネルギー量}
\section{よ}
\item{弱い力}{}
{不安定原子核が$\beta$崩壊する際に働く力。素粒子標準理論では、すべてのフェルミオンの間でWボソンやZボソンという粒子を交換されることで力が作用しあうと考える。}
\item{4中心2電子分子軌道積分}{}
{2電子クーロン反発積分のうち4つの分子軌道中心を持つ2電子クーロン反発積分のこと。}
\section{ら}
\item{ラジカル}{}
{電子が対になっていないことで不安定になっている化学物質}
\item{ランチョス法}{}
{エルミート行列を三重対角化する手法。数値計算において再帰計算による効率的な演算が可能であることから、固有方程式の解法等でよく用いられる。}
\item{乱流境界層}{}
{乱流で構成された境界層(粘性を有する流体中において粘性の影響を強く受ける領域で、一般には物体表面に見られる) 乱流境界層では流体の渦運動により運動量やエネルギーの交換が強く行われる。このため、壁面近傍の流体へ運動量が供給されるので層流境界層よりも剥離しにくいが、壁面付近で急激に減少する速度分布を持つため摩擦抗力が大きい。}
\item{乱流スケール}{}
{乱流における渦の大きさ}
\section{り}
\item{リオーダリング}{}
{計算順序の並べ替えのこと.計算処理を速くしたり,並列処理ができるように依存関係をなくすために行われる.}
\item{リガンド結合}{}
{受容体に特異的に結合する物質(リガンド)が結合すること}
\item{リボゾーム}{}
{細胞内の構造体で、遺伝情報からタンパク質へと変換する機構である翻訳が行われる場である。}
\item{量子色力学}{}
{クォークとグルーオンの力学(強い力の力学)を量子力学的に記述する理論。素粒子標準理論の一部をなす。 (QCD: Quantum Chromodynamics)}
\item{量子数射影法}{}
{量子多体計算において、変分計算によって得られた波動関数はハミルトニアンがもともと持っている対称性を自発的に破っていることが多い。この波動関数に射影演算子を作用させることによって、本来保つべき対称性を回復させる方法。}
\item{リラクサー}{}
{特殊な強誘電体。誘電率の周波数依存性に特徴がある。}
\item{臨界終点}{}
{相図において一次相転移が終結し、熱力学変数が連続的に変化するようになる(クロスオーバー)へと移行する点}
\item{隣接通信}{}
{並列計算で領域分割法を用いる時に、隣り合う分割された領域間でデータの授受を行う通信の事。}
\section{る}
\item{ループ}{}
{ファインマン図形において現れるループ構造。量子補正が高次になるにつれループの数が増える。}
\item{ルシフェラーゼ}{}
{蛍の蛍光タンパク質}
\item{ルミノシティ}{}
{ビーム衝突型加速器実験において、ルミノシティ=単位時間あたりに起こる反応の回数÷断面積で定義される。}
\section{れ}
\item{レアイベント探索アルゴリズム}{}
{通常のシミュレーションでは滅多に発生しないが、科学的に重要な事象を探索するためのアルゴリズム。例えば分子動力学シミュレーションでは高い活性化障壁をもった化学反応はなかなか発生しない。}
\item{レイトレーシング}{}
{光線追跡法.コンピュータグラフィクスの画像生成手法の一つで,光が反射屈折する物理現象を模倣し,画像を作成する.}
\item{レオロジー}{}
{物質の流動と変形を取り扱う学問。}
\item{レジスタブロッキング}{}
{CPU内のレジスタ上になるべくデータを集められるようにするためのコーディングテクニック。これにより命令実行効率が向上する。}
\item{レプリカ法}{}
{同じ原子から構成されるシステムを複数用意し、それぞれシミュレーションの条件を変えながら、シミュレーションを行う方法。条件パラメータをある一定の法則に従って交換しながら実行するレプリカ交換法などがある。}
\section{ろ}
\item{ローカルインジェクション}{}
{外部からガラス細管などで分子を脳の中の特定の領域に注入することさらに電圧を同時に付加することで特定領域の細胞内に分子を注入するローカルエレクトロポレーションなども存在する。}
\item{ローカルファイルシステム}{}
{並列計算機の各計算ノードから独立して参照されるファイルシステム。各計算ノードで個別に使われるファイルを一時的に保存する場所として使われ、他の計算ノードからは参照できないためグローバルファイルシステムと比較して利便性に欠ける。一方でグローバルファイルシステムと比較して特に大規模なシステムにおいて高性能を達成しやすい構成である。}
\end{用語集}
