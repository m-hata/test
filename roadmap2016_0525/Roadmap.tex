\documentclass[11pt,a4paper]{jsbook}
\usepackage{here}
\usepackage{chapterbib}
\usepackage[dvipdfmx]{graphicx,color,hyperref}
\usepackage{pxjahyper}
\hypersetup{%
  pdftitle={計算科学ロードマップ},
  pdfauthor={今後のHPCIを使った計算科学発展のための検討会},
  bookmarksnumbered=true,
  colorlinks=true,
  citecolor=black,
  filecolor=black,
  linkcolor=black,
  urlcolor=black,
}

\usepackage{color}
\usepackage{colortbl}
\usepackage{array}
\usepackage{booktabs}
\usepackage{longtable}
\usepackage{threeparttablex}
\usepackage{wrapfig}

\input macro.tex

\setlength{\textwidth}{\fullwidth}
\setlength{\evensidemargin}{\oddsidemargin}

% 参考文献スタイル
%\newcommand{\rmbibstyle}{jplain}
\newcommand{\rmbibstyle}{junsrt}

\begin{document}
\title{計算科学ロードマップ\\
〜 大規模並列計算によるイノベーションの目指す\\
社会貢献・科学的成果 〜}
\author{今後のHPCIを使った計算科学発展のための検討会}
\date{平成28年X月}
\maketitle

\tableofcontents

\mainmatter

\chapter{序論}
\section{はじめに}
\label{sec:はじめに}

 % はじめに
\section{分野連携・大規模実験施設}
\label{sec:分野連携} % 分野連携・大規模実験施設
\section{将来実現しうる大規模計算機}
\label{sec:将来のHPC}
 % 将来実現しうる大規模計算機

\chapter{各計算科学分野の課題}
\section{素粒子・原子核}
\label{sec:素粒子原子核}

\subsection{分野の概要}

\subsection{長期目標と社会貢献}

\subsection{課題とその解決に必要な計算手法・アプリケーション}

\subsection{ロードマップ}

\subsection{必要な計算機資源}


% 参考文献
\nocite{*}
\bibliographystyle{\rmbibstyle}
\bibliography{2-1}
 % 素粒子・原子核
\begin{要求性能表}{物質科学}{%
  \begin{TableNotes}
    \item[1] 100~1000くらいのアレイジョブを想定
    \item[2] 整数演算がメイン
    \item[3] メモリ使用量はMPIプロセス数に比例し最大使用量を記載した
    \item[4] 電子状態計算の要求性のは第一原理計算のxTAPP、古典MDはMODYLAS、I/Oの部分は東大渡辺による短距離古典MD(東大渡辺さん)のデータをベースに概算
    \item[5] 電子状態計算の要求性のは第一原理計算のxTAPP、古典MDはMODYLAS、I/Oの部分は東大渡辺による短距離古典MD(東大渡辺さん)のデータをベースに概算
    \item[6] 電子状態計算の要求性のは第一原理計算のxTAPP、古典MDはMODYLAS、I/Oの部分は東大渡辺による短距離古典MD(東大渡辺さん)のデータをベースに概算
    \item[7] アレイジョブでノード間通信なし
    \item[8] PHASEの1/10の規模であることから、同時実行はこの表では想定していない
    \item[9] 1ノード100TFLOPS, 10000ノード並列を仮定
  \end{TableNotes}
}{\insertTableNotes}
%
次世代先端デバイス &
100 & 100 & 1.2 & 10 & 96 & 10 & 350000 &
第一原理計算RSDFT(擬ポテンシャル法、実空間基底) &
原子数:10万
\\ \midrule
%
次世代先端デバイス &
100 & 100 & 2 & 15 & 60 & 100 & $2.2\times 10^{6}$ &
第一原理計算PHASE(擬ポテンシャル、平面波基底、O(N\nobreak${}^{3}$)法) &
原子数:1万 100MDを同時実行
\\ \midrule
%
次世代先端デバイス &
100 & 100 & 2 & 15 & 60 & 100 & $2.2\times 10^{6}$ &
第一原理計算xTAPP(擬ポテンシャル、平面波基底、O(N\nobreak${}^{3}$)法) &
原子数:1万 100MDを同時実行
\\ \midrule
%
次世代先端デバイス &
100 & 20 & 5 & 10 & 240 & 10 & 860000 &
第一原理計算CONQUEST(密度行列、最適化によるO(N)法) &
原子数:1億 2fsの時間刻みで25000でナノ秒オーダーを想定 計算時間は要注意。時間ステップ数10\nobreak${}^{4}$。電子材料の電子状態計算・手法1と同じ計算だが、こちらは個々のケースを高速に計算する必要があり、ネットワーク性能をより要求する。ストレージ量の違いは出力頻度の違いによる。
\\ \midrule
%
光・電子デバイス\tnote{1} &
1000 & 10 & 10 & 0.1 & 1 & 100 & 360000 &
高精度分子軌道法 &
2万基底、100万求積点
\\ \midrule
%
分子機能 &
300 & 18 & 4 & 0.0001 & 15 & 10 & 160000 &
大規模分子軌道法 &
原子数:1万
\\ \midrule
%
分子機能(タンパク質の電子状態) &
1.1 & 0.19 & 1 & 0.001 & 1 & 100 & 400 &
フラグメント分子軌道法 &
数百残基のタンパク質、数千万次元の密行列の固有値問題
\\ \midrule
%
熱交換デバイスの安全性向上・特性解析 &
20 & 6.4 & 51 & 44 & 24 & 10 & 17000 &
短距離古典分子動力学 &
粒子数:4000億
\\ \midrule
%
分子機能と物質変換 &
1000 & 100 & 2 & 1000 & 150 & 10 & $5.4\times 10^{6}$ &
長距離古典分子動力学 &
原子数:10億
\\ \midrule
%
光・電子材料 &
600 & 200 & 200 & 33 & 14 & 10 & 300000 &
ナノ構造体電子・電磁波ダイナミクス法 &
原子数:96万, 時間は1ステップあたり1秒で計算量は0.63EFLOP。これを50000ステップでおよそ14時間
\\ \midrule
%
強相関電子系の機能解明\tnote{2} &
3 & 390 & 10 & 10 & 10 & 100 & 11000 &
クラスターアルゴリズム量子モンテカルロ法 &
原子数:1億
\\ \midrule
%
強相関電子系の機能解明\tnote{3} &
1000 & 300 & 0.2 &  & 8 & 100 & $2.9\times 10^{6}$ &
変分モンテカルロ法 &
原子数1万
\\ \midrule
%
物質・エネルギー変換\tnote{4} &
500 & 50 & 0.008 & 6.4 & 2.8 & 10 & 50000 &
量子分子動力学法 &
100レプリカ、100万ステップ
\\ \midrule
%
物質・エネルギー変換\tnote{5} &
690 & 69 & 2 & 3.2 & 300 & 10 & $7.4\times 10^{6}$ &
化学反応動力学・量子分子動力学法(分子軌道計算またはQM/MM) &
QM1000原子、10000レプリカ、10000step, MM100,000原子(roadmap)
\\ \midrule
%
物質・エネルギー変換\tnote{6} &
410 & 41 & 0.02 & 0.05 & 20 & 10 & 300000 &
化学反応動力学・量子分子動力学法(第一原理計算)\ &
数万レプリカ
\\ \midrule
%
分子構造・分子機能\tnote{7} &
1000 & 0.5 & 0.04 &  & 24 & 1 & 86000 &
分子動力学法(feramによるリラクサー強誘電体の誘電率の周波数依存) &
512x512x512
\\ \midrule
%
新物質探索 &
4100 & 41 & 20 &  & 0.5 & 1 & 7400 &
クラスター展開法(第一原理計算) &
原子数:1万, 100イオン配置の同時実行
\\ \midrule
%
新材料\tnote{8} &
0.1 & 0.02 & 0.00012 &  & 24 & 10000 & 86000 &
第一原理計算(凍結フォノン法) &
原子数:1万
\\ \midrule
%
強相関電子系の機能解明 &
82 & 130 & 82 & 41 & 42 & 10 & 120000 &
厳密対角化(ランチョス法) &
54サイトのスピン系(Sz=0)
\\ \midrule
%
新物質探索\tnote{9} &
690 & 1600 & 1.5 & 20 & 24 & 20 & $1.2\times 10^{6}$ &
フェーズフィールド法 &
10\nobreak${}^{13}$空間メッシュ、10\nobreak${}^{7}$時間ステップ
\end{要求性能表}
 % ナノサイエンス・デバイス
\section{エネルギー・材料}
\label{sec:エネルギー材料}

\subsection{分野の概要}

\subsection{長期目標と社会貢献}

\subsection{課題とその解決に必要な計算手法・アプリケーション}

\subsection{ロードマップ}

\subsection{必要な計算機資源}


% 参考文献
\nocite{*}
\bibliographystyle{\rmbibstyle}
\bibliography{2-2}
 % エネルギー・材料
\section{生命科学}
\label{sec:生命科学}

\subsection{分野の概要}

\subsection{長期目標と社会貢献}

\subsection{課題とその解決に必要な計算手法・アプリケーション}

\subsection{ロードマップ}

\subsection{必要な計算機資源}


% 参考文献
\nocite{*}
\bibliographystyle{\rmbibstyle}
\bibliography{2-4}
 % 生命科学
\section{創薬・医療}
\label{sec:創薬医療}

\subsection{分野の概要}

\subsection{長期目標と社会貢献}

\subsection{課題とその解決に必要な計算手法・アプリケーション}

\subsection{ロードマップ}

\subsection{必要な計算機資源}


% 参考文献
\nocite{*}
\bibliographystyle{\rmbibstyle}
\bibliography{2-5}
 % 創薬・医療
\section{設計・製造}
\label{sec:設計製造}

\subsection{分野の概要}

\subsection{長期目標と社会貢献}

\subsection{課題とその解決に必要な計算手法・アプリケーション}

\subsection{ロードマップ}

\subsection{必要な計算機資源}


% 参考文献
\nocite{*}
\bibliographystyle{\rmbibstyle}
\bibliography{2-6}
 % 設計・製造
\section{社会科学}
\label{sec:社会科学}

\subsection{分野の概要}

\subsection{長期目標と社会貢献}

\subsection{課題とその解決に必要な計算手法・アプリケーション}

\subsection{ロードマップ}

\subsection{必要な計算機資源}


% 参考文献
\nocite{*}
\bibliographystyle{\rmbibstyle}
\bibliography{2-7}
 % 社会科学
\section{脳科学・人工知能}
\label{sec:脳人工知能}

\subsection{分野の概要}

\subsection{長期目標と社会貢献}

\subsection{課題とその解決に必要な計算手法・アプリケーション}

\subsection{ロードマップ}

\subsection{必要な計算機資源}


% 参考文献
\nocite{*}
\bibliographystyle{\rmbibstyle}
\bibliography{2-8}
 % 脳科学・人工知能
\section{地震・津波}
\label{sec:地震津波}

\subsection{分野の概要}

\subsection{長期目標と社会貢献}

\subsection{課題とその解決に必要な計算手法・アプリケーション}

\subsection{ロードマップ}

\subsection{必要な計算機資源}


% 参考文献
\nocite{*}
\bibliographystyle{\rmbibstyle}
\bibliography{2-9}
 % 地震・津波
\section{気象・気候}
\label{sec:気象気候}

\subsection{分野の概要}

\subsection{長期目標と社会貢献}

\subsection{課題とその解決に必要な計算手法・アプリケーション}

\subsection{ロードマップ}

\subsection{必要な計算機資源}


% 参考文献
\nocite{*}
\bibliographystyle{\rmbibstyle}
\bibliography{2-10}
 % 気象・気候
\section{宇宙・天文}
\label{sec:宇宙天文}

\subsection{分野の概要}

\subsection{長期目標と社会貢献}

\subsection{課題とその解決に必要な計算手法・アプリケーション}

\subsection{ロードマップ}

\subsection{必要な計算機資源}


% 参考文献
\nocite{*}
\bibliographystyle{\rmbibstyle}
\bibliography{2-11}
 % 宇宙・天文

\chapter{アプリケーションの分類}
\section{ミニアプリとの対応}
\label{sec:ミニアプリ} % ミニアプリとの対応
\section{計算機アーキテクチャから見たアプリケーションの分類}
\label{seq:アーキからのアプリ分類} % 計算機アーキテクチャから見たアプリケーションの分類

\chapter{各課題の詳細}
\section{素粒子・原子核}
\label{sec:素粒子原子核_詳細}
 % 素粒子・原子核
\begin{要求性能表}{物質科学}{%
  \begin{TableNotes}
    \item[1] 100~1000くらいのアレイジョブを想定
    \item[2] 整数演算がメイン
    \item[3] メモリ使用量はMPIプロセス数に比例し最大使用量を記載した
    \item[4] 電子状態計算の要求性のは第一原理計算のxTAPP、古典MDはMODYLAS、I/Oの部分は東大渡辺による短距離古典MD(東大渡辺さん)のデータをベースに概算
    \item[5] 電子状態計算の要求性のは第一原理計算のxTAPP、古典MDはMODYLAS、I/Oの部分は東大渡辺による短距離古典MD(東大渡辺さん)のデータをベースに概算
    \item[6] 電子状態計算の要求性のは第一原理計算のxTAPP、古典MDはMODYLAS、I/Oの部分は東大渡辺による短距離古典MD(東大渡辺さん)のデータをベースに概算
    \item[7] アレイジョブでノード間通信なし
    \item[8] PHASEの1/10の規模であることから、同時実行はこの表では想定していない
    \item[9] 1ノード100TFLOPS, 10000ノード並列を仮定
  \end{TableNotes}
}{\insertTableNotes}
%
次世代先端デバイス &
100 & 100 & 1.2 & 10 & 96 & 10 & 350000 &
第一原理計算RSDFT(擬ポテンシャル法、実空間基底) &
原子数:10万
\\ \midrule
%
次世代先端デバイス &
100 & 100 & 2 & 15 & 60 & 100 & $2.2\times 10^{6}$ &
第一原理計算PHASE(擬ポテンシャル、平面波基底、O(N\nobreak${}^{3}$)法) &
原子数:1万 100MDを同時実行
\\ \midrule
%
次世代先端デバイス &
100 & 100 & 2 & 15 & 60 & 100 & $2.2\times 10^{6}$ &
第一原理計算xTAPP(擬ポテンシャル、平面波基底、O(N\nobreak${}^{3}$)法) &
原子数:1万 100MDを同時実行
\\ \midrule
%
次世代先端デバイス &
100 & 20 & 5 & 10 & 240 & 10 & 860000 &
第一原理計算CONQUEST(密度行列、最適化によるO(N)法) &
原子数:1億 2fsの時間刻みで25000でナノ秒オーダーを想定 計算時間は要注意。時間ステップ数10\nobreak${}^{4}$。電子材料の電子状態計算・手法1と同じ計算だが、こちらは個々のケースを高速に計算する必要があり、ネットワーク性能をより要求する。ストレージ量の違いは出力頻度の違いによる。
\\ \midrule
%
光・電子デバイス\tnote{1} &
1000 & 10 & 10 & 0.1 & 1 & 100 & 360000 &
高精度分子軌道法 &
2万基底、100万求積点
\\ \midrule
%
分子機能 &
300 & 18 & 4 & 0.0001 & 15 & 10 & 160000 &
大規模分子軌道法 &
原子数:1万
\\ \midrule
%
分子機能(タンパク質の電子状態) &
1.1 & 0.19 & 1 & 0.001 & 1 & 100 & 400 &
フラグメント分子軌道法 &
数百残基のタンパク質、数千万次元の密行列の固有値問題
\\ \midrule
%
熱交換デバイスの安全性向上・特性解析 &
20 & 6.4 & 51 & 44 & 24 & 10 & 17000 &
短距離古典分子動力学 &
粒子数:4000億
\\ \midrule
%
分子機能と物質変換 &
1000 & 100 & 2 & 1000 & 150 & 10 & $5.4\times 10^{6}$ &
長距離古典分子動力学 &
原子数:10億
\\ \midrule
%
光・電子材料 &
600 & 200 & 200 & 33 & 14 & 10 & 300000 &
ナノ構造体電子・電磁波ダイナミクス法 &
原子数:96万, 時間は1ステップあたり1秒で計算量は0.63EFLOP。これを50000ステップでおよそ14時間
\\ \midrule
%
強相関電子系の機能解明\tnote{2} &
3 & 390 & 10 & 10 & 10 & 100 & 11000 &
クラスターアルゴリズム量子モンテカルロ法 &
原子数:1億
\\ \midrule
%
強相関電子系の機能解明\tnote{3} &
1000 & 300 & 0.2 &  & 8 & 100 & $2.9\times 10^{6}$ &
変分モンテカルロ法 &
原子数1万
\\ \midrule
%
物質・エネルギー変換\tnote{4} &
500 & 50 & 0.008 & 6.4 & 2.8 & 10 & 50000 &
量子分子動力学法 &
100レプリカ、100万ステップ
\\ \midrule
%
物質・エネルギー変換\tnote{5} &
690 & 69 & 2 & 3.2 & 300 & 10 & $7.4\times 10^{6}$ &
化学反応動力学・量子分子動力学法(分子軌道計算またはQM/MM) &
QM1000原子、10000レプリカ、10000step, MM100,000原子(roadmap)
\\ \midrule
%
物質・エネルギー変換\tnote{6} &
410 & 41 & 0.02 & 0.05 & 20 & 10 & 300000 &
化学反応動力学・量子分子動力学法(第一原理計算)\ &
数万レプリカ
\\ \midrule
%
分子構造・分子機能\tnote{7} &
1000 & 0.5 & 0.04 &  & 24 & 1 & 86000 &
分子動力学法(feramによるリラクサー強誘電体の誘電率の周波数依存) &
512x512x512
\\ \midrule
%
新物質探索 &
4100 & 41 & 20 &  & 0.5 & 1 & 7400 &
クラスター展開法(第一原理計算) &
原子数:1万, 100イオン配置の同時実行
\\ \midrule
%
新材料\tnote{8} &
0.1 & 0.02 & 0.00012 &  & 24 & 10000 & 86000 &
第一原理計算(凍結フォノン法) &
原子数:1万
\\ \midrule
%
強相関電子系の機能解明 &
82 & 130 & 82 & 41 & 42 & 10 & 120000 &
厳密対角化(ランチョス法) &
54サイトのスピン系(Sz=0)
\\ \midrule
%
新物質探索\tnote{9} &
690 & 1600 & 1.5 & 20 & 24 & 20 & $1.2\times 10^{6}$ &
フェーズフィールド法 &
10\nobreak${}^{13}$空間メッシュ、10\nobreak${}^{7}$時間ステップ
\end{要求性能表}
 % ナノサイエンス・デバイス
\section{エネルギー・材料}
\label{sec:エネルギー材料_詳細}
 % エネルギー・材料
\section{生命科学}
\label{sec:生命科学_詳細} % 生命科学
\section{創薬・医療}
\label{sec:創薬医療_詳細}
 % 創薬・医療
\section{設計・製造}
\label{sec:設計製造_詳細} % 設計・製造
\section{社会科学}
\label{sec:社会科学_詳細}
 % 社会科学
\section{脳科学・人工知能}
\label{sec:脳人工知能_詳細} % 脳科学・人工知能
\section{地震・津波}
\label{sec:地震津波_詳細}
 % 地震・津波
\section{気象・気候}
\label{sec:気象気候_詳細}
 % 気象・気候
\section{宇宙・天文}
\label{sec:宇宙天文_詳細}
 % 宇宙・天文

\chapter{おわりに}
%\input 5.tex

\appendix
\chapter{用語集}
\input glossary.tex

\chapter{執筆者一覧}
\input authors.tex

\end{document}
